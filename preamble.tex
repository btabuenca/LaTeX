%%%%%%%
%Language
%%%%%%%
\usepackage[british]{babel}
\usepackage[applemac]{inputenc}
\usepackage[T1]{fontenc}

%%%%%%%
%Figures
%%%%%%%
\usepackage{graphicx}								%include pictures, diagrams, or drawings
\usepackage[small, bf, up, hang]{caption}					%customization of captions
%\usepackage[percent]{overpic}						%enables Latex constructs over graphics
\usepackage{adjustbox}								% Rotating figures



\usepackage{changepage}							% Indenting text

%\usepackage{caption}
\usepackage{subfig}
%\usepackage{tabularx}
%\usepackage{float}

%%%%%%%
%Special characters
%%%%%%%
\usepackage{pifont}


%%%%%%%
%Tables
%%%%%%%
\usepackage{booktabs}								%enhance the quality of your tables by new line commands replacing \hline and \cline
\usepackage{multirow}
\usepackage{rotating}								%pro?vi?des the side?ways environ?ment; rotate a whole table by using the side?waysta?ble instead of the table environment
\usepackage{threeparttable}							%package facilitates tables with titles (captions) and notes
\usepackage{longtable}

%%%%%%%
%Math
%%%%%%%
\usepackage{amsmath}								%com?mands and environ?ments for maths
\usepackage{nicefrac}								%nice?frac com?mand, which crea?tes a frac?tion with dia?go?nal frac?tion line
\usepackage{units}									%com?mands unit[2.3]{N} and unitfrac[4.3]{Nm}{s} for nicely set units.
\usepackage{textgreek}
\usepackage{textcomp}

%%%%%%%
%Bibliography
%%%%%%%
% The following 4 lines were added by btw
%\usepackage[american]{babel}
%\usepackage{csquotes}
%\usepackage[style=apa,backend=biber]{biblatex}
%\DeclareLanguageMapping{american}{american-apa}

\usepackage{natbib} %careful with this one, it makes the bibliography work. Include this to use the apalike style!
\usepackage{url}

%%%%%%%
%Miscallenious
%%%%%%%
\usepackage{lscape}

\usepackage{microtype}

\usepackage{xspace}

\usepackage{helvet}

\usepackage{lipsum}

\usepackage[usenames,dvipsnames,table]{xcolor}							%coloring columns, rows, single entries, and lines in many ways

\usepackage{fancyhdr}								%loads 'fancy' page style to allow customisation of header and footer
\fancyhf{}
\fancyhead[LE]{\nouppercase{\leftmark}}
\fancyhead[RO]{\nouppercase{\rightmark}}
\fancyfoot[LE,RO]{\thepage}

\fancypagestyle{plain}{%redefining plain pagestyle
	\fancyfoot[LE,RO]{\thepage} %prints the page number on the right side of the header
	\renewcommand{\headrulewidth}{0pt}
	\fancyhead{}
}

\usepackage{parskip}
\usepackage{paralist}								%provides several new list environments designed to be typeset within paragraphs or in a very compact look

\usepackage[section]{placeins}							%implicit \FloatBarrier for tables and figures to be used at the beginning of each section

\usepackage{cleveref}								%automatically determines the type of cross-reference and the context in which it's used

\newcommand{\head}[1]{\textnormal{\textbf{#1}}}

\makeatletter
\renewcommand\part{%
  \if@openright
    \cleardoublepage
  \else
    \clearpage
  \fi
  \thispagestyle{empty}%   % Original �plain� replaced by �emptyx
  \if@twocolumn
    \onecolumn
    \@tempswatrue
  \else
    \@tempswafalse
  \fi
  \null\vfil
  \secdef\@part\@spart}
\makeatother

% Thicker line
\makeatletter
\newcommand{\thickhline}{%
    \noalign {\ifnum 0=`}\fi \hrule height 1pt
    \futurelet \reserved@a \@xhline
}
\newcolumntype{"}{@{\hskip\tabcolsep\vrule width 1pt\hskip\tabcolsep}}
\makeatother
