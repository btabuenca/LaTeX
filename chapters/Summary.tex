\chapter*{Summary\footnote{This chapter incorporates abstracts, discussions, and conclusions from several publications.}\markboth{Summary}{}}
\addcontentsline{toc}{chapter}{Summary}

%\vfill
%\clearpage

NOT YET FINISHED

This thesis presented the results of the conducted research and development of ubiqui-tous technology to support lifelong learners. The reported results are structured into three parts: theoretical foundations, formative studies, and empirical findings.

In the first part we aimed at getting grasp of how, when, what, and where do lifelong learners learn using their mobile devices as a key channel to support them across con-texts. Therefore, chapter 1 outlined these behaviors as a starting point to identify in which contexts lifelong leaners might need support. OERs are one of the main sources for lifelong learners to obtain relevant content. Thus, chapter 2 offered a vision of the state-of-art of content repositories� features for mobile devices from two different perspectives. From the content providers side, the results obtained from a survey per-formed on 23 educational repository owners shed light on what is their current and expected support to facilitate access from mobile devices. From the content user side, existing tools were explored and key features were identified based on a review of scientific literature. This section is completed with chapter 3 that reviews previous work on NFC technology as a relevant technology to scaffold learning ecologies con-necting content to physical spaces with mobile devices.

In the second part, a set of tools were designed and piloted as an approach to facilitate lifelong learners to stop and explore available resources that might be relevant to his/her own learning interests, and to identify in which settings the user performs better. Therefore, chapter 3 proposes sampling of learning preferences on mobile devices as key benchmarks for lifelong learners to become aware on which learning task better suits which context. A mobile application for sampling of learning experi-ences is piloted to identify which are the preferences from lifelong learners when accomplishing a learning activity.

In traditional education settings, teachers normally create content which is later on delivered to the students. However, lifelong learners do not always have teachers and recur to existing resources stored repositories to reuse and remix them towards ac-complishing their own learning purposes. Mobile authoring tools enable to foster universal access to educational resources capturing the context in-situ and in-time. Indeed, authoring educational resources in a mobile context is an authentic experience where authors can link learning with their own daily life activities and reflections. Thus, chapter 5 identifies the main barriers for mobile authoring and 10 key short-comings are identified based on a review of existing mobile authoring tools. Inspired on these findings, a mobile authoring tool to cover these gaps is designed, evaluated.

Lifelong learners' activities are scattered along the day, in different locations and making use of multiple devices. Most of the times they have to merge learning, work and everyday life making difficult to have an account on how much time is devoted to learning activities and learning goals. Indeed, learning experiences are disrupted and there is a lack of solutions to integrate daily life activities and learning in the same process. On the other hand, smartphones are becoming a universal learning device facilitating new tools and ways of interaction that can be smoothly embedded into daily life. Chapter 6 presents the NFC LearnTracker, a mobile tool proposing the user to introspect his autobiography as a learner to identify successful physical learning environments, mark them with sensor tags, bind them to self-defined learning goals, keep track of the time invested on each goal with a natural interface, and monitor the learning analytics. This work implies a suitable tool for lifelong learners to bind scat-tered activities keeping them in a continuing learning flow. The NFC LearnTracker is released under open access licence with the aim to foster adaptation to further com-munities as well as to facilitate the extension to the increasing number of sensor and NFC tags existent in the market.

A fundamental objective of human-computer interaction research is to make sys-tems that are seamlessly integrated into daily life activities. Hence, the challenge for technology enhanced-learning research is not only to make information available to people at any time, at any place, and in any form, but specifically to say the �right� thing at the �right� time in the �right� way. On the other hand, the proliferation of sensor technology is facilitating the scaffolding and customization of smart learning environments. This manuscript presents an ecology of resources comprising NFC, BLE and Arduino technology, orchestrated in the context of a learning environment to provide smoothly integrated feedback via ambient displays. This ecology is proposed as a suitable solution for self-regulation, providing support for setting learning goals, setting aside time to learn, tracking study time and monitoring the progress. Herby, the ecology is described and intriguing research questions are introduced.



Experimental settings

Nowadays, smartphone users are constantly receiving notifications from applica-tions that provide feedback, as reminders, recommendations or announcements. Nev-ertheless, there is little research on the effects of mobile notifications to foster meta-learning. This paper explores the effectiveness of mobile notifications to foster reflec-tion on meta-learning by presenting the results of two studies: 1) a formative study with 37 secondary school students offering a daily reflection and reporting exercise about their learning experience during the day; 2) an experiment involving 60 adults to read an eBook on energy-efficient driving for one hour. During that time the partic-ipants received mobile notifications inviting them to reflect inaction. On the one hand, the results from the first study show that students do not have a habit to see themselves as learners and to develop a "professional" awareness about their daily activity at work/school. On the other hand, the second study explores the effects of different notification types on knowledge gain and motivation. Results envision a higher knowledge gain and motivation for the group assigned with the least complex interactions with mobile devices during the reflection exercise. Finally, these results are discussed and important research questions for future research on mobile notifications are raised.

This longitudinal study explores the effects of tracking and monitoring time devoted to learning with a mobile tool on self-regulated learning. Graduate students (N=36) from three different online courses used their own mobile devices to track how much time they devoted to learning over a period of two to four months. Repeated measures of the Online Self-Regulated Learning Questionnaire (OSLQ) and Validity and Relia-bility of Time Management Questionnaire (VRTMQ) were taken along the courses. Our findings reveal positive effects of tracking time on skills/knowledge in time man-agement. Variations in the channel, content and timing of the mobile notifications for selfreflection are investigated and time-logging patterns are described with regard to the impact of the notifications offered to students. These results not only provide evi-dence of the benefits of recording learning time, but also suggest relevant cues on how mobile notifications should be designed and prompted towards self-regulation of students in online courses.



