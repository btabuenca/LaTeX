% To-Dos:
%

\chapter*{Summary\footnote{This chapter incorporates abstracts, discussions, and conclusions from several publications.}\markboth{Summary}{}}
\addcontentsline{toc}{chapter}{Summary}

%\vfill
%\clearpage

This thesis presented the results of the conducted research and development of ambient learning displays. The reported results are structured into three parts: theoretical foundations, formative studies, and empirical findings. An elaborated conceptual framework and an extensive literature review were used to explore the research field and lay the foundation for further research. \textbf{Chapter 1} outlined the vision of ambient learning displays � enabling learners to view, access, and interact with contextualised digital content presented in an ambient way. This vision was based on a detailed exploration of the characteristics of ubiquitous learning and a deduction of informational, interactional, and instructional aspects to focus on. To provide a theoretical foundation for the following research, relevant research findings, models, design dimensions, and taxonomies were examined. The result was a conceptual framework that defined ambient learning displays. The framework consists of parts dedicated to user and context data acquisition, channelling of information, and delivery of contextualised information framed in a learning process. 

The first part of the conducted literature review, presented in \textbf{Chapter 2}, depicted characteristics, classified prototypical designs, and shed light on the actual use of the covered ambient displays, their application context and addressed domains as well as the type of studies conducted, including the used methodologies and evaluation approaches to measure their effectiveness and impact. The results showed that the acquisition and delivery of information through ambient displays were in line with the presented conceptual framework. The means to channel the information and the framing into a learning process needed further investigation. The literature review continued in \textbf{Chapter 3}. This second part focused on the actual use of ambient displays in a learning context. The goal was to assess ambient displays with an explicit or implicit learning purpose and the classification of respective prototypes on the basis of an extended classification framework, including the informational and interactional design of the prototypes, research objectives and results of the reported empirical studies, as well as deducible instructional characteristics. The results exposed that the explicit use of ambient displays for learning was not a prominent research topic, although implicitly ambient displays were already used to support learning activities fostering situational awareness by exploiting feedback. The mapping of the corresponding prototypes with the introduced classification framework led to first general design implications, taking into account instructional characteristics.

Several formative studies informed the theoretical work as well as the design and development from different perspectives. \textbf{Chapter 4} described an explorative study conducted to inform the research of ambient learning displays. By asking domain experts the used concept mapping approach identified the major educational problems that can be addressed by mobile learning and clustered these problems into domain concepts that contribute to a definition of mobile learning. The main domain concepts identified were �access to learning� and �contextual learning�. This reflected the claim on mobile learning to enable learning across context, facilitating and exploiting the mobility of the learners. Although the study targeted on the mobile learning domain, the results were in a broader view also considered valuable for the ubiquitous learning domain and thus for the conducted research. 

The results of two projects that informed the design and development of ambient learning displays were presented. The first project, presented in \textbf{Chapter 5}, elaborated and developed an infrastructure that supports energy conservation at the workplace. The purpose was to make energy consumption data visible and accessible to employees by providing dynamic situated consumption feedback. The presented results showed the general interest in the topic and indicated the effectiveness of the introduced means towards the conservation of energy. The second project, presented in \textbf{Chapter 6}, implemented a pervasive game to increase the environmental awareness and pro-environmental behaviour at the workplace. In relation to the previous project the purpose was to go beyond increasing awareness and providing personalised information and instead focus on the potential of a pervasive game to increase knowledge, pro-environmental consciousness, and last but not least change consumption behaviour. The results showed that incentive mechanisms are less important than challenging game components that involve employees in proposing solutions for energy conservation at the workplace. 

\textbf{Chapter 7} then described a lecture series that summarised the theoretical foundations. Furthermore the chapter reported on a participatory design study conducted in the course of the lectures with the goal to inform and ease the design process of ambient displays for learning. The presented results showed a variety of usable ambient display types, possible learning scenarios, and specific design proposals towards ambient learning displays. 

Following up the theoretical work and the formative studies, respective ambient learning display prototypes were then evaluated in empirical studies. The first empirical study into the research and development of ambient learning displays was presented in \mbox{\textbf{Chapter 8}}. The first part of the study reported an intervention to initiate environmental learning and facilitate pro-environmental behaviour. The purpose was to examine the impact of ambient learning displays on energy consumption and conservation at the workplace, more specifically the evaluation of learning outcome and behaviour change. The results did not provide clear evidence that the design of the displays influences the learning outcome or that the displays lead to pro-environmental behaviour change. Nevertheless the sole deployment of the display prototypes eased the comprehension of the information provided and lowered the need for additional information. Furthermore, the results provided insights and revealed several challenges for future research. The second part of the study, presented in \textbf{Chapter 9}, then focused on the interaction between ambient displays and users. The main purpose was to examine the general user attention towards ambient displays as well as the influence of different display designs. The study combined non-intrusive evaluation techniques as a quantitative approach to measure user attention with qualitative measurement of user perception and comprehension. The results showed a high degree of user interest in the displays over time, but did not provide clear evidence that the design of the displays influences the user attention. Nevertheless the combination of quantitative and qualitative measurement provided a more holistic view on user attention. Several guidelines for an effective attention-aware display design were derived.

Finally, the second empirical study into the research and development of ambient learning displays was presented in \textbf{Chapter 10}. The study reported an intervention to investigate previously identified research challenges on the evaluation and use of ambient displays in a learning context with the objective to gain insights into the interplay between display design, user attention, and knowledge acquisition. The results provided evidence that an attention-aware display design attracts and retains user attention more effectively and that there is a positive relation between knowledge gain and user attention. Furthermore, the design significantly facilitated the acquisition of knowledge.

The thesis concluded with a \textbf{General Discussion} reviewing all reported results and their practical implications, general limitations of the conducted research, as well as future research perspectives. Overall the conducted research and development revealed that ambient displays could be designed and implemented to fulfil a given purpose successfully, possibly also for learning. Once implemented the known long-term effects as well as the contextual factors that influence the display�s efficiency need further investigation. In the dawning age of ubiquitous computing, ambient displays represent a technological concept with great potential for learning. The presented vision of ambient learning displays highlighted the challenges and explored the possibilities that lie in the convergence of mobile and ubiquitous learning in combination with the utilisation of contextualised digital content as valuable resources to support learning. The empirical findings delivered new scientific insights into the authentic learning support in informal and non- formal learning situations. The conducted research was mainly limited regarding the chosen application domains, the prototypical ambient display designs, and the occurring tensions when evaluating between lab and field settings. Towards ambient learning displays still some work needs to be done, wherein this thesis can be taken as basis and inspiration to go beyond.