\chapter*{Summary\footnote{This chapter incorporates abstracts, discussions, and conclusions from several publications.}\markboth{Summary}{}}
\addcontentsline{toc}{chapter}{Summary}

Lifelong learners learn in scattered moments throughout the day and the week, and they have to balance learning activities with work, family and leisure activities. In this thesis, we have explored how, when, what and where lifelong learners are learning by using their mobile devices as a key channel to support them across contexts. 
The research conducted for this thesis has been guided by the following objectives:

\begin{enumerate}
\item Understanding personal learning ecologies of lifelong learners.
\item Facilitate ubiquitous access to digital learning resources.
\item Link learning activities with everyday life activities and the physical world objects.
\item Develop artifacts and software prototypes that allow the implementation of educational designs in which the learners connect informal learning experiences with formal learning activities in and across contexts.
\item Identify barriers for digital competence in lifelong learners. 
\item Foster reflective practice on meta-learning. 
\end{enumerate}

In this thesis, smartphones support lifelong learners across contexts (e.g. at home, at work, commuting, having a break) since they are always carried around with the learner. \textbf{Chapter 1} identifies learning patterns in daily life environments as a starting point to understand how lifelong learners learn using their mobile devices, and in which contexts they might need support. 

Offering educational resources openly is one of the key policies to facilitate universal access for lifelong learners. In \textbf{chapter 2} we have observed the current support from content repositories to mobile devices, identifying best practices and suitable architectures for the implementation of software that might facilitate universal access to content stored in repositories. Additionally, smartphones enable learners to author educational resources not only providing channels to share, remix or re-contextualize these, but also capturing the context in-situ and in-time. As a further matter, authoring educational resources in a mobile context is an authentic experience where authors can create learning resources inspired on their own daily life activities and reflections. 

A strength of lifelong learners is their motivation to learn and their continuous career development. Hence, they continuously try to understand and improve the way they learn identifying best contexts (i.e. time, location, technology) and embedding learning activities into daily life activity. In this thesis, we offered lifelong learners to introspect their learning habits, and annotate their reflections on personal mobile devices as key benchmarks to become aware of successful learning environments and to identify their learning preferences in context (i.e. preferred time, location, resources). A classification framework for sampling of experiences on mobile devices is presented (See figure \ref{fig:Esm_fig1}) and instantiated in \textbf{chapter 4}. Later on, this framework is again evaluated in experimental settings (\textbf{chapters 6, 8, 9}). 

In \textbf{chapter 6} we identify the lack of support for learning activities across locations, devices, and environments, as well as, the necessity to link learning activities with everyday life activities as key challenges for lifelong learners. There is very little research on how to link the different everyday contexts of lifelong learners and their learning activities in these different settings. As reported in \textbf{chapter 1}, lifelong learners recur to certain physical spaces (e.g. sofa, desktop) or moments (e.g. commercial breaks on TV, waiting times, commuting) to accomplish their learning activities. The NFC-LearnTracker presented in this thesis was recognized as a relevant tool with the potential identify those successful learning environments, to manage self-defined learning goals, to keep track of the time devoted to each learning goal, and to monitor their progress with learning analytics visualizations.

In this thesis we aimed at supporting learners to understand the way they can better learn using technology.  We focused on two specific key competences for lifelong learning \citep{EuropeanCommission2007}, namely, �learning to learn� and �digital competence�.

On the one hand side, �learning to learn� implies the learner to understand how to make the best use of own skills and available resources towards the accomplishment of self-defined learning goals. Therefore, in \textbf{chapter 8} we instantiate again the classification framework for sampling experiences (figure \ref{fig:Esm_fig1}) by prompting learners compact and structured notifications suggesting to examine and evaluate their own learning. These notifications were prompted as structured and repeated introspective episodes, offered during the course of the learning action as well as after the learning action, to make learning visible. The results show that students do not have a habit to see themselves as learners and to develop a "professional" awareness about their daily activity at work/school. The effectiveness of mobile notifications to foster reflective practice on meta-learning is further researched in the longitudinal study reported in \textbf{chapter 9}. In this study we analysed the timing and the content of the notifications. The findings conclude that notifications pushed at fixed times of the day moderate positively the measure of time management. These results are consistent with the answers reported by the learners regarding their timing preference in which, notifications at 10h were preferred over notifications at 20h as well as over notifications randomized in time. Another reason to argue on these results might be that students prefer notifications that persuade them to (pre-)�plan ahead� their learning day, rather than (post-)�look backward� their learning day or (in-action) �plan� at any moment of the day. Observing at the content of the notifications, the findings in the experiment suggest that notifications containing learning analytics and tips for self-regulation influence positively the skill of time management. More specifically, notifications containing learning analytics resulted in slightly higher scores. Likewise, \textbf{chapter 9} provides an interesting perspective on how learners devote their time to learn throughout the day, along the week and in long term.

On the other hand, �digital competence� involves the confident and critical use of technology to learn, work, and communicate in personal and social life. Nowadays, technology is constantly shaping the way we learn, thus designing interfaces that are easy to use and to integrate into daily life activity becomes increasingly relevant for lifelong learners. This assumption is reinforced by the results reported in the second study from \textbf{chapter 8}, in which the group assigned with the least complex interactions on mobile devices achieved higher knowledge and motivation scores.

The results from the survey in \textbf{chapter 1} show that lifelong learners reported their living room (and sitting in the sofa) as the most suitable environment to watch videos using their mobile devices for learning purposes. In \textbf{chapter 3}, we orchestrate an ecology of digital resources in the context of this specific environment to facilitate casting of video content stored in the cloud by lowering the complexity of interactions with devices (i.e. reducing the time to start the learning activity, reducing the number of clicks to access the learning content), as well as improving the resolution casting HD quality as an alternative to small-sized definition on mobiles, tablet or laptops. 

In this research, we investigated different ways to foster the competence of �learning to learn�, using SMSs and mobile chart visualizations as channels to provide guidance (external feedback). The differentiation among external and internal feedback is crucial if one investigates the effects of feedback viewing the process of knowledge acquisition as a self-regulated learning process \citep{Narciss2007}. Hence, in the last stage of the research (\textbf{chapter 7}) we extended the NFC-LearnTracker implementing a ecology of resources for lifelong learners to customize their internal feedback based on own occasional learning priorities and contingencies.

