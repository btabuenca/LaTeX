% To-Dos:

\chapter{Tap it again, Sam: harmonizing personal environments towards lifelong learning} % Write in your own chapter title

%\begin{quote}
%\textbf{Abstract:} With a focus on the situated support of informal and non-formal
%learning scenarios in ubiquitous learning environments, the presented paper outlines the authors� vision of ambient learning displays � enabling
%learners to view, access, and interact with contextualised digital content
%presented in an ambient way. The vision is based on a detailed exploration of
%the characteristics of ubiquitous learning and a deduction of informational,
%interactional, and instructional aspects to focus on. Towards the vision essential
%research questions and objectives as well as a conceptual framework that
%acquires, channels, and delivers the information framed in the learning process
%are presented. To deliver scientific insights into the authentic learning support
%in informal and non-formal learning situations and to provide suggestions for
%the future design of ambient systems for learning the presented paper concludes with a
%research agenda proposing a research project including a discussion of related
%issues and challenges.
%\end{quote}
\vfill
The first part of the thesis looks into the theoretical foundations for the following research. This chapter starts with outlining the vision of ambient learning displays and elaborating on a conceptual framework. Relevant research findings, models, design dimensions, and taxonomies are examined to deduce informational, interactional, and instructional aspects to focus on. The resulting conceptual framework consists of parts dedicated to user and context data acquisition, channelling of information, and delivery of contextualised information framed in a learning process. The chapter concludes with a research agenda.
\vspace{3em}

This chapter is published as: B�rner, D., Kalz, M., and Specht, M. (2011). Thinking outside the box � A vision on ambient learning displays. \textit{International Journal of Technology Enhanced Learning}, 3(6), 627�642.
\clearpage

\section{Introduction}

\section{Literature Review In Nfc Technology For Learning}
\subsection{Formal Education}
\subsection{Guided Tours And Fieldtrips}
\subsection{User Identification}
\subsection{Activity Recognition \& Life Logging}
\subsection{Smart Home}
\subsection{Support Of Disabled People}
\subsection{Payment Systems And Simulations}
\subsection{Logistics \& Object Identification}
\subsection{Results Of The Review}

\section{A Seamless Ecology To Learn From Videos}
\subsection{The NFC MediaPlayer}
\subsection{Implementation}


\section{Discussion and Conclusions}






