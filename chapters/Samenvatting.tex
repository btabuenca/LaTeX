\chapter*{Samenvatting\footnote{Dit hoofdstuk bevat samenvattingen van, discussies over en conclusies uit verschillende publicaties.}\markboth{Samenvatting}{}}

\addcontentsline{toc}{chapter}{Samenvatting}

Leven-lang-lerenden leren op verschillende tijdstippen gedurende de dag en de week. Ze moeten voor hun leeractiviteiten een balans vinden tussen activiteiten op hun werk, binnen hun gezin en in hun vrije tijd. In dit proefschrift hebben we onderzocht hoe, wanneer, wat en waar een leven-lang-lerende leert met behulp van hun mobiele apparaten, die als een belangrijk kanaal dienen om hen binnen verschillende contexten te ondersteunen. Het onderzoek dat gedaan werd voor dit proefschrift is gebaseerd op de volgende doelstellingen:


\begin{enumerate}
\item het begrijpen van persoonlijke ecologie�n op het gebied van leren voor leven-lang-lerenden
\item faciliteren van alomtegenwoordige toegang tot digitale leermiddelen
\item het koppelen van leeractiviteiten aan de activiteiten in het dagelijks leven en objecten in de fysieke wereld
\item het ontwikkelen van artefacten en software prototypes die de implementatie toestaan van educatieve ontwerpen waarin de leerlingen informele leerervaringen met formele leeractiviteiten verbinden binnen en over contexten heen
\item het identificeren van obstakels voor digitale competentie in levenslang leren
\item het aanmoedigen van reflectieve oefeningen voor meta-leren.

\end{enumerate}

In dit proefschrift smartphones ondersteunen leven-lang-lerenden te ondersteunen over contexten heen (figuur \ref{survey_fig:6}: bijvoorbeeld thuis, op het werk, tijdens het woon-werkverkeer, tijdens een pauze), omdat de lerende deze altijd bij zich heeft. \textbf{Hoofdstuk 1} identificeert leerpatronen in het dagelijks leven als een startpunt om te begrijpen hoe een leven-lang-lerende leert met behulp van mobiele apparaten, en in welke contexten hij misschien ondersteuning nodig heeft.

Het openlijk aanbieden van educatieve middelen is een van de belangrijkste beleidsmaatregelen om de algemene toegang tot een leven lang leren te vergemakkelijken. In \textbf{hoofdstuk 2} hebben we gekeken naar de huidige ondersteuning voor mobiele apparaten in content repositories. Dit werd gedaan om gangbare praktijken en geschikte architecturen voor de implementatie van de software te kunnen vaststellen, om zo de algemene toegang tot inhoud - opgeslagen in die repositories - te vergemakkelijken.

Daarenboven maken smartphones niet alleen de creatie van educatieve middelen mogelijk via het verstrekken van kanalen om deze te delen, te remixen of opnieuw te contextualiseren, maar is het ook mogelijk om context vast te leggen in-situ en in-time. Verder is het cre�eren van educatieve middelen in een mobiele context een authentieke ervaring waarbij auteurs leerbronnen kunnen maken, ge�nspireerd door hun eigen dagelijkse activiteiten en reflecties. 

Een sterk punt voor de leven-lang-lerende is zijn intrinsieke motivatie om te leren en zijn voortdurende loopbaanontwikkeling. Vandaar dat hij de manier waarop hij leert voortdurend probeert te begrijpen en te verbeteren, door het identificeren van de beste contexten (d.w.z. tijd, locatie, technologie), en leeractiviteiten in te passen in dagelijkse gewoonten. In dit proefschrift hebben we leven-lang-lerenden de mogelijkheid geboden om hun leergewoontes te bekijken en aantekeningen te maken van hun reflecties op hun eigen mobiele apparaten als belangrijkste ijkpunten om zich bewust te worden van succesvolle leeromgevingen en hun persoonlijke voorkeuren over de manier van leren binnen contexten (d.w.z. de gewenste tijd, locatie, middelen). Er wordt een classificatiekader voor het verzamelen van ervaringen met mobiele apparaten gepresenteerd (zie figuur \ref{fig:Esm_fig1}) en ge�nstantieerd in \textbf{hoofdstuk 4}. Later wordt dit kader opnieuw ge�valueerd in experimentele settings (\textbf{hoofdstukken 6, 8, 9}).


In \textbf{hoofdstuk 6} identificeren we het gebrek aan ondersteuning voor leeractiviteiten wat betreft locaties, apparaten en omgevingen, alsook de noodzaak om leeractiviteiten te koppelen aan activiteiten in het dagelijks leven als belangrijkste uitdagingen voor een leven-lang-lerende. Er is zeer weinig onderzoek gedaan naar hoe de verschillende alledaagse contexten van een leven-lang-lerende en zijn leeractiviteiten in deze verschillende settings te koppelen. Zoals beschreven in \textbf{hoofdstuk 1} gebruikt een leven-lang-lerende steeds weer bepaalde fysieke ruimtes (bijvoorbeeld een bank, desktop) of momenten (bijvoorbeeld reclameblokken op tv, wachttijden, woon-werkverkeer) om zijn leeractiviteiten te voltooien. De NFC-LearnTracker, die gepresenteerd wordt in dit proefschrift, werd erkend als een relevant hulpmiddel met de potentie om die succesvolle leeromgevingen te herkennen, om zelf gedefinieerde leerdoelen te beheren, om bij te houden hoeveel tijd aan elk leerdoel besteed werd en om vooruitgang te monitoren met learning analytics visualisaties.

In dit proefschrift hebben we geprobeerd om lerenden te ondersteunen bij het begrijpen van de manier waarop ze beter kunnen leren met behulp van technologie. We hebben ons gericht op twee specifieke kerncompetenties voor een leven lang leren \citep{EuropeanCommission2007}, namelijk, �leren leren� en �digitale competentie�.

Aan de ene kant zet �leren leren� de lerende ertoe aan om te begrijpen hoe het beste gebruik te maken van de eigen vaardigheden en de beschikbare middelen voor het bereiken van zelfgedefinieerde leerdoelen. In \textbf{hoofdstuk 8} instanti�ren we weer het classificatiekader voor het verzamelen van ervaringen (figuur \ref{fig:Esm_fig1}) door leerlingen compacte en gestructureerde meldingen aan te bieden om hun eigen leerproces te onderzoeken en te evalueren. Deze notificaties werden aangeboden als gestructureerde en herhalende introspectieve gebeurtenissen, zowel tijdens de leeractiviteit, alsook na de leeractiviteit, om het leren zichtbaar te maken. De resultaten tonen aan dat studenten niet de gewoonte hebben om zichzelf te zien als lerenden en geen �professioneel� bewustzijn over hun dagelijkse bezigheden op het werk/de school ontwikkelen.

De effectiviteit van mobiele notificaties om een reflectieve toepassing op meta-leren te bevorderen wordt verder onderzocht in het longitudinale onderzoek uit \textbf{hoofdstuk 9}. In dit onderzoek hebben we de timing van notificaties geanalyseerd. Uit de resultaten kan worden geconcludeerd dat notificaties op een dagelijks vast tijdstip een positief tijdmanagement opleveren. Deze resultaten komen overeen met het antwoord dat lerenden gaven op de vraag over hun voorkeur voor tijd. Leerlingen verkozen notificaties om 10.00 uur boven de notificaties om 20.00 uur, alsook de notificaties die op een willekeurig tijdstip werden verzonden. Een andere reden om over deze resultaten te argumenteren zou kunnen zijn, dat studenten eerder de voorkeur geven aan notificaties die hen ertoe aanzetten om de dag waarop ze leren (vooraf)�vooruit te plannen� dan dat ze (achteraf) �terugkijken� op de dag of (in actie) �plannen� op elk moment van de dag. Als we de inhoud van de notificaties bekijken, suggereren de resultaten van het experiment dat notificaties die learning analytics bevatten en tips voor zelfregulatie het beheersen van tijdmanagement positief be�nvloeden. Met name notificaties die learning analytics bevatten scoorden iets hoger. Zo biedt ook \textbf{hoofdstuk 9} een interessant perspectief op hoe lerenden hun tijd om te leren door de dag heen, gedurende de week en over een grote tijdsspanne gebruiken.

Aan de andere kant heeft 'digitale competentie' betrekking op het vertrouwen in en het kritisch gebruik van technologie om te leren, te werken en te communiceren op persoonlijk en sociaal vlak. Tegenwoordig geeft technologie voortdurend vorm aan de manier waarop we leren en zo wordt het ontwerpen van interfaces die gemakkelijk zijn in het gebruik en gemakkelijk te integreren zijn in het dagelijks leven steeds relevanter voor de leven-lang-lerende. Deze aanname wordt versterkt door de resultaten uit het tweede onderzoek in \textbf{hoofdstuk 8}, waarin de groep die de minst complexe interacties met mobiele apparaten kreeg toegewezen meer kennis verwierf en hogere motivatiescores had. 

De resultaten van het onderzoek uit \textbf{hoofdstuk 1} tonen dat leven-lang-lerenden aangaven, dat de woonkamer (en het zitten op de bank) de meest geschikte omgeving was om naar video�s te kijken terwijl ze hun mobiele apparaat gebruikten voor leerdoeleinden. In \textbf{hoofdstuk 3} hebben we een ecologie van digitale bronnen samengesteld binnen de context van deze specifieke omgeving om het uitzenden van video-inhoud die opgeslagen is in de cloud te ondersteunen door het verlagen van de complexiteit van interacties met apparaten (d.w.z. de tijd inkorten om een leeractiviteit op te starten, het aantal klikken verminderen om toegang tot de leerinhoud te krijgen) evenals het verbeteren van de resolutie voor het uitzenden in HD-kwaliteit als een alternatief voor de lagere resoluties bij mobiele telefoons, tablet of laptops.

In dit onderzoek hebben we verschillende manieren onderzocht om de competentie �leren leren� te ondersteunen met gebruik van sms en mobiele grafische visualisatie als methoden om begeleiding te geven. De differentiatie tussen externe en interne feedback is cruciaal als men de effecten onderzoekt van feedback met het oog op het proces van kenniswerving als een zelfregulerende leerproces \citep{Narciss2007}.

Daarom hebben we in de laatste fase van het onderzoek (\textbf{hoofdstuk 7}) de NFC-LearnTracker uitgebreid door een ecologie van bronnen te implementeren voor leven-lang-lerenden, zodat ze in staat zijn om hun interne feedback aan te passen, gebaseerd op hun eigen, niet gestructureerde leerbehoeften en onvoorziene gebeurtenissen.
