% To-Dos:
%

\chapter*{Samenvatting\footnote{Dit hoofdstuk bevat samenvattingen van, discussies over en conclusies uit verschillende publicaties.}\markboth{Samenvatting}{}}
\addcontentsline{toc}{chapter}{Samenvatting}

%\vfill
%\clearpage

In dit proefschrift worden de resultaten gepresenteerd van het uitgevoerde onderzoek en de ontwikkeling van ambient learning displays. De resultaten worden in drie delen gerapporteerd: theoretische grondslagen, formatieve studies en empirische bevindingen. Een uitgewerkt conceptueel kader en een uitgebreid literatuuronderzoek werden gebruikt om het veldonderzoek te verkennen en deze vormden de basis voor verder onderzoek. \textbf{Hoofdstuk 1} schetst de visie op ambient learning displays - het voor leerlingen mogelijk maken om gecontextualiseerd digitale content die op een ambiente manier aangeboden wordt, te bekijken, zich toegang ertoe te verschaffen en ermee te interageren. Deze visie is gebaseerd op een gedetailleerde verkenning van de kenmerken van ubiquitous learning en een gevolgtrekking van informatieve, interactionele en educatieve aspecten waar we ons op richten. Om te zorgen voor een theoretische grondslag voor het volgende onderzoek, werden relevante onderzoeksresultaten, modellen, ontwerpdimensies en taxonomie�n onderzocht. Het resultaat was een conceptueel kader dat ambient learning displays omschrijft. Het kader bestaat uit delen die zich toelegden op de gebruiker en op context data-acquisitie, op het kanaliseren van informatie en op het doorgeven van gecontextualiseerde informatie, ingekaderd in een leerproces. 

Het eerste deel van de uitgevoerde literatuurstudie, gepresenteerd in \textbf{Hoofdstuk 2}, toont kenmerken en geclassificeerde prototypische ontwerpen en belicht het daadwerkelijke gebruik van de beschreven ambient displays, de context waarin ze toepasbaar zijn en de aangesproken domeinen alsook het type van uitgevoerde studies, met inbegrip van de gebruikte methoden en evaluatiebenaderingen om hun doeltreffendheid en het effect te meten. De resultaten toonden aan dat het verkrijgen en het doorgeven van informatie via ambient displays in lijn was met het gepresenteerde conceptuele kader. De middelen om de informatie en het inkaderen in een leerproces te kanaliseren behoeven verder onderzoek. Het literatuuronderzoek werd voortgezet in \textbf{Hoofdstuk 3}. Dit tweede deel was gericht op het werkelijke gebruik van ambient displays in een leeromgeving. Het doel was om ambient displays te beoordelen met een expliciet of impliciet leerdoel en het indelen van de respectievelijke prototypes op grond van een uitgebreid classificatiekader met informatieve en interactionele vormgeving van de prototypes, onderzoeksdoelstellingen en resultaten van de gerapporteerde empirische studies, evenals afleidbare educatieve eigenschappen. Uit de resultaten bleek dat het expliciete gebruik van ambient displays voor leeractiviteiten geen prominent onderzoeksonderwerp was, terwijl ambient displays impliciet al werden gebruikt ter ondersteuning van leeractiviteiten ter stimulering van situationeel bewustzijn door het benutten van feedback. Het vergelijken van de overeenkomende prototypes met het ingevoerde classificatiekader heeft geleid tot de eerste algemene gevolgtrekking met betrekking tot het ontwerp, rekening houdend met educatieve kenmerken.

Verschillende formatieve studies vormden input voor het theoretische werk, alsmede voor het ontwerp en de ontwikkeling vanuit verschillende perspectieven. \textbf{Hoofdstuk 4} beschrijft een exploratieve studie die werd uitgevoerd om input te geven aan het onderzoek van ambient learning displays. Door domeinexperts te vragen naar de gebruikte aanpak voor concept mapping werden de voornaamste educatieve problemen ge�dentificeerd, welke door mobile learning opgelost kunnen worden en daarna werden deze problemen gebundeld in domeinconcepten die bijdragen aan een definitie van mobile learning. De belangrijkste domeinconcepten die werden ge�dentificeerd, waren "access to learning" en "contextual learning". Deze reflecteerden de potentie om het mogelijk te maken om contextoverstijgend te leren met behulp van mobile learning en het vergemakkelijken en het benutten van de mobiliteit van lerenden. Hoewel de studie was gericht op het domein van mobile learning, werden de resultaten in een breder perspectief ook waardevol geacht voor een ubiquitous learning omgeving en daarmee voor het verrichte onderzoek. 

De resultaten van twee projecten, die input gaven aan het ontwerp en de ontwikkeling van ambient learning displays, werden gepresenteerd. Het eerste project, gepresenteerd in \textbf{Hoofdstuk 5}, onderzocht en ontwikkelde een infrastructuur die energiebesparing ondersteunt op de werkplek. Het doel was gegevens over het energieverbruik zichtbaar en toegankelijk te maken voor de medewerkers door middel van het verschaffen van dynamische feedback in situ over het verbruik. De gepresenteerde resultaten toonden de algemene interesse in het onderwerp en gaven de effectiviteit aan van de ge�ntroduceerde middelen met betrekking tot energiebesparing. In het tweede project, gepresenteerd in \textbf{Hoofdstuk 6}, werd een pervasive game ge�mplementeerd om het milieubewustzijn en het pro-milieugedrag op de werkplek te vergroten. Ten opzichte van het vorige project was het doel verder te gaan dan het verhogen van het bewustzijn en het verstrekken van persoonlijke informatie en in plaats daarvan zich te richten op de mogelijkheden van een pervasive game om kennis en pro-milieubewustzijn te vergroten en niet in de laatste plaats om het consumptiegedrag te veranderen. De resultaten toonden dat stimulerende mechanismen minder belangrijk zijn dan uitdagende spelonderdelen waarbij medewerkers betrokken zijn bij het voordragen van oplossingen voor energiebesparing op de werkplek.

\textbf{Hoofdstuk 7} beschrijft een lezingenreeks die de theoretische grondslagen samenvat. Verder wordt in het hoofdstuk gerapporteerd over een participatieve ontwerpstudie, uitgevoerd tijdens de lezingen, met als doel om input te geven aan het ontwerpproces voor ambient learning displays en het te vergemakkelijken. De gepresenteerde resultaten toonden verschillende bruikbare typen van ambient display, mogelijke leerscenario's en specifieke ontwerpvoorstellen richting ambient learning displays.

Volgend op het theoretische werk en de formatieve studies werden vervolgens de desbetreffende ambient learning displays ge�valueerd in empirische studies. De eerste empirische studie naar het onderzoek en de ontwikkeling van ambient learning displays werd gepresenteerd in \textbf{Hoofdstuk 8}. Het eerste deel van de studie rapporteerde een interventie om milieuleer te initi�ren en pro-milieugedrag te vergemakkelijken. Het doel was om de invloed van ambient learning displays te onderzoeken op energieverbruik en besparing op de werkplek, meer bepaald de evaluatie van leerresultaat en gedragsverandering. De resultaten hebben geen duidelijk bewijs geleverd dat het ontwerp van de displays het leerresultaat be�nvloedt of dat de displays leiden tot pro-milieu gedragsverandering. Toch maakte enkel de inzet van de display prototypes het begrip van de verstrekte informatie gemakkelijker en verminderde de behoefte aan aanvullende informatie. Bovendien verschaften de resultaten inzichten voor verder onderzoek en boden een aantal uitdagingen. Het tweede deel van de studie, beschreven in \mbox{\textbf{Hoofdstuk 9}}, richtte zich vervolgens op de interactie tussen ambient displays en gebruikers. Het belangrijkste doel was om de algemene aandacht van de gebruiker voor ambient displays te onderzoeken, alsook de invloed van verschillende displayontwerpen. De studie combineerde niet-intrusieve evaluatietechnieken als een kwantitatieve benadering om de aandacht van de gebruiker te meten met een kwalitatieve meting van de perceptie en het begrip van de gebruiker. De resultaten toonden een hoge mate van interesse van de gebruiker voor de displays op de lange duur, maar gaven geen duidelijk bewijs dat het ontwerp van de displays de aandacht van de gebruiker be�nvloedt. Toch geeft de combinatie van kwantitatieve en kwalitatieve meting een meer holistische kijk op de aandacht van de gebruiker. Er werden verschillende richtlijnen gevonden voor een display-ontwerp, dat effectief de aandacht bewust bevordert.

Tenslotte werd de tweede empirische studie in het onderzoek en de ontwikkeling van ambient learning displays gepresenteerd in \textbf{Hoofdstuk 10}. De studie rapporteerde een interventie om eerder geconstateerde onderzoeksuitdagingen over de evaluatie en het gebruik van ambient displays in een leercontext te onderzoeken met als doel om inzicht te krijgen in de wisselwerking tussen display-ontwerp, aandacht van de gebruiker en kennisverwerving. De resultaten leverden het bewijs dat een aandachtsbewust display-ontwerp effectiever de aandacht trekt en behoudt en dat er een positieve relatie is tussen het verwerven van kennis en aandacht van de gebruiker. Bovendien vergemakkelijkt het ontwerp het verwerven van kennis aanzienlijk.

Het proefschrift wordt afgesloten met een \textbf{Algemene Discussie} die alle gerapporteerde resultaten en de praktische gevolgen daarvan, algemene beperkingen van het uitgevoerde onderzoek, alsmede de toekomstig onderzoeksperspectieven beoordeelt. Globaal bekeken maakt het gevoerde onderzoek en de ontwikkeling duidelijk, dat ambient displays kunnen worden ontworpen en ge�mplementeerd om met succes te voldoen aan een bepaald doel, eventueel ook voor het leren. Eenmaal ge�mplementeerd is er verder onderzoek nodig naar de bekende langetermijneffecten en de contextuele factoren die de efficiency van de display be�nvloeden. In het aanbrekende tijdperk van ubiquitous computing representeren ambient displays een technologisch concept met een groot potentieel voor het leren. De gepresenteerde visie over ambient learning displays benadrukt de uitdagingen en onderzoekt de mogelijkheden die samen gaan met mobile and ubiquitous learning in combinatie met het gebruik van gecontextualiseerde digitale content als waardevol middel ter ondersteuning van het leren. De empirische bevindingen leverden nieuwe wetenschappelijke inzichten in de authentieke ondersteuning bij het leren in informele en non-formele leersituaties. Het uitgevoerde onderzoek werd voornamelijk beperkt op het gebied van de gekozen toepassingsdomeinen, de prototypische ambient display-ontwerpen en de optredende spanningen bij het evalueren tussen het lab en het veldwerk. Omtrent ambient learning displays moet nog wat werk verricht worden, waarbij dit proefschrift als basis en inspiratie genomen kan worden om verder mee te werken.