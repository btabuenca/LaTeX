\chapter{Time Will Tell: The role of mobile learning analytics in self-regulation} 


\vfill
Nowadays, smartphone users bla bla
\vspace{3em}

This chapter has been submitted as: 
Tabuenca, B., Kalz, M., \& Specht, M. (2015). (Submitted) Time Will Tell: The role of mobile learning analytics in self-regulation. \em Internet and Higher Education \em

\clearpage

\section{Introduction}

\subsection{Self-regulation with mobile technology}

\subsubsection{Psychology of notifications}

\subsubsection{Learning analytics}

\subsubsection{Seamless learning}

\section{Method}

\subsection{Participants}

\subsection{Materials}

\subsubsection{LearnTracker Backend}

\textbf{Database model}


\textbf{Webservices}


\subsubsection{Mobile clients}

\textbf{LearnTracker for Android}


\textbf{Multiplatform web interface}

\subsubsection{Notifications and SMS broadcasting tool}

\subsection{Design of the experiment}

\subsection{Measure instruments}

\subsubsection{Self-regulation}

\subsubsection{Validity and reliability of time management}

\subsubsection{Time patterns}

\subsubsection{Complexity of the mobile tool}

\subsection{Procedure}

\subsection{Data analysis}

\section{Results}

\subsection{Impact of logging/monitoring time in self-regulation}

\subsection{Impact of the timing in the notifications in self-regulation}


\subsubsection{Patterns sampling study time}

\subsubsection{How do students log their time}

\subsubsection{Correlation between time-logs and performance}


\subsection{Impact from the content of the notifications in self-regulation}


\subsubsection{Preference in content and channels}


\subsection{Usability of the tool}



\section{Discussion}


\subsection{Interpretation of the results}

\subsection{Limitations}


\subsection{Significance of the study}

