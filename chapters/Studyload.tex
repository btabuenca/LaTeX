% To-Dos:

\chapter{Time will tell: Self-regulation of time with mobile learning analytics. IN PROGRESS} % Write in your own chapter title

%\begin{quote}
%\textbf{Abstract:} With a focus on the situated support of informal and non-formal
%learning scenarios in ubiquitous learning environments, the presented paper outlines the authors� vision of ambient learning displays � enabling
%learners to view, access, and interact with contextualised digital content
%presented in an ambient way. The vision is based on a detailed exploration of
%the characteristics of ubiquitous learning and a deduction of informational,
%interactional, and instructional aspects to focus on. Towards the vision essential
%research questions and objectives as well as a conceptual framework that
%acquires, channels, and delivers the information framed in the learning process
%are presented. To deliver scientific insights into the authentic learning support
%in informal and non-formal learning situations and to provide suggestions for
%the future design of ambient systems for learning the presented paper concludes with a
%research agenda proposing a research project including a discussion of related
%issues and challenges.
%\end{quote}
\vfill
Nowadays, smartphone users bla bla
\vspace{3em}

This chapter is published as: 
Tabuenca, B., Kalz, M., \& Specht, M. (2015). (Submitted) Time will tell: Self-regulation of time with mobile learning analytics. \em IEEE Transactions on Learning Technologies (TLT) \em, (Special Issue on Seamless, Ubiquitous, and Contextual Learning), 1�12. doi:10.1109/TLT.2014.2383611

\clearpage

\section{Introduction}


\section{Study load}
\subsection{Method}
\subsubsection{Participants}


\subsubsection{Materials}

\subsubsection{Design}

\subsubsection{Procedure}
\subsection{Results}
   
\section{Discussion and conclusions}
