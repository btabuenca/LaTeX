\chapter*{General Discussion\footnote{This Chapter incorporates discussions and conclusions from several publications.}\markboth{General Discussion}{}}
\addcontentsline{toc}{part}{General Discussion}

%\vfill
%\clearpage

The research reported in this thesis aimed at identifying daily practices of adult lifelong learners and how they can be supported with technology in and across contexts. Therefore in a first step everyday patterns of lifelong learners were identified and mapped on a model of context. From the perspective of the content, best practices to access educational resources from mobile devices were pinpointed as well as potential approaches to bind digital educational resources to physical spaces. Based on these findings, in a second step a set of prototypes to support lifelong learners were designed constituting a substantial base of tools and knowledge towards its evaluation in experimental settings. In a third step, these tools are evolved with the aim to facilitate lifelong learners to manage everyday life and learning activities in a more enjoyable and efficient way.

\section*{Main findings}
This research has been guided by the following objectives:
\begin{itemize}
\item Understanding personal learning ecologies of lifelong learners. The results from the survey in \textbf{Chapter 1} show different patterns on how lifelong learners use their mobile devices to learn. Portable computers are perceived as the most commonly used device to learn. On the other hand, the results show that the simple fact of owning a smartphone makes the user more constantly motivated to learn throughout the day, probably due to their perception of replacing "lost time� into perceived "productive time". Indeed, participants reported that they usually multitask while accomplishing learning activities with their mobile devices, specially  \em listening \em contents in which they reported to devote more time and in longer time slots. Regarding their preferred physical space to learn using mobile devices, there seems to be an association between the learning activity being performed (i.e.  \em reading, listening, writing, watching \em ) and the concrete location where it takes place (i.e. living room, kitchen, bathroom). Additionally, the reports from the participants in the survey indicate that learning experiences are disrupted, whereas �finding a suitable time slot to learn during the day� is perceived as the most frequently found barrier. In this first step of the research we also observed they way in which mobile users dispatch incoming notifications identifying two behaviors to take into account when designing mobile applications for learning: check continuously; check on-alert (e.g. vibration, tone, icon).

\item Facilitate ubiquitous access to digital learning resources. \textbf{Chapter 2} reviews the state-of-art of learning repositories featuring mobile access to digital resources. This review explored different implementations accessing content indexed by GPS coordinates, compass orientation, gesture recognition via accelerometer, radio frequency identification, infrared, visual code scanning, Bluetooth approximation, text-marker recognition or image recognition. This review also looks at the granularity of the resources suggesting that easy to create and to consume contents on mobile devices, are those whose unit of construction is text, audio or video (minimum granularity). Deepening more into technical implementations, the lessons learned out of the review pinpoint to server architectures based on RESTful web-services featuring an API, as well as HTML5 client, as suitable software implementations to facilitate content accessibility and scalability for lifelong learners. In \textbf{Chapter 1}, the participants in the survey identified the sofa as the most preferred personal environment to accomplish any learning activity using their mobile devices. In \textbf{Chapter 3}, we support lifelong learners in this context piloting the NFC-MediaPlayer, an ecology of resources comprising NFC tags, a multimedia casting tool and a learning content referratory, to cast videos from a mobile device to an HDMI display interpreting the commands (i.e. play, pause, forward, cast) of the NFC tag that is tapped with the mobile device. This novel ecology facilitates seamless access to video contents reducing the time to start the learning activity, reducing the number of clicks to access the learning content (to zero clicks), and casting in High-Definition quality, learning contents that are normally casted in small-sized screens on mobiles, tablets or laptops. Additionally, we share the �know how� releasing the source code under open license to foster its scalability to further software communities.

\item Linking learning activities with everyday life activities and the physical world objects. In \textbf{Chapter 1}, NFC technology was identified as a prominent trend in the field of mobile technology in the last years. This technology is particularly relevant for lifelong learners due to its natural interaction reducing the complexity to accomplish predefined actions to zero-clicks, and consequently facilitating a smooth integration of learning activities into everyday life. \textbf{Chapter 3} reviewed scientific literature in which NFC has been used with learning purposes and potential uses for learners are classified. As a consequence of this review, �activity recognition and life logging� were identified as relevant fields to link learning activities with everyday life activities. Indeed, lifelong learners' activities are scattered along the day and in different locations making difficult to manage learning goals and to have an account of how much time is devoted to each one of them. Nevertheless, the results reported in \textbf{Chapter 1} show that lifelong learners frequently recur to preferred learning environments throughout their learning journey (physical spaces, devices, or moments). Sometimes the physical environment itself may relate to learning activities in other cases only temporal patterns might be of relevance while the physical location is not crucial. Therefore, it is important for lifelong learners to introspect their autobiography as a learner, in order to identify those successful physical learning environments, to bind them to self-defined learning goals, to keep track of the time devoted to each goal, to monitor their progress and consequently to understand how does he/she learn. The NFC-LearnTracker presented in \textbf{Chapter 6} supports lifelong learners in those functions, providing a suitable tool to foster awareness on preferred learning environments, to manage learning goals, to have an accurate account on time devoted to learn, and to analyze personal learning patterns.

\item Foster awareness of available resources, locations and moments to learn in everyday life. In \textbf{Chapter 1}, lifelong learners participating in the survey reported that their learning experiences are disrupted arguing the difficulty to �find a suitable time slot to learn during the day� as the most popular reason. In \textbf{Chapter 4} we proposed sampling of learning experiences on mobile devices as key benchmarks for lifelong learners to become aware on which learning task suits in which context, to set realistic goals and to set aside time to learn on a regular basis. Thus, a classification framework for sampling of experiences on mobile devices is presented as an approach to explore variations to deliver, dispatch and answer notifications in context. These framework is not only useful when prompting notifications to second persons (e.g. from teacher to student), they are significantly more powerful and accurate when they are self-reports due to the fact that the lifelong learner has an intrinsic motivation to know himself as a learner and consequently to provide truthful reports having only himself as a reference. This framework is first instantiated in a small-scale study piloting the Experience Sampling Method (ESM) app in which lifelong learners were able to reflect and record their learning preferences in context, through the use of different features from their smartphones. Later on, this classifications framework is again instituted in \textbf{chapters 6, 8 and 9}.

\item Develop artifacts and software prototypes that will allow the implementation of educational designs in which the learners connect informal learning experiences with formal learning activities in and across contexts. This thesis comprises the development of six different software tools shared in open source, namely, NFC-MediaPlayer ( \textbf{Chapter 3} ), ESM app ( \textbf{Chapter 4} ), Mobile Authoring Tool for ARLearn ( \textbf{Chapter 5} ), NFC-LearnTracker ( \textbf{Chapter 6} ), smart ecology of resources for effective time management ( \textbf{Chapter 7} ) and the LearnTracker framework ( \textbf{Chapter 9} ). These tools facilitate connecting informal and formal learning experiences featuring: a) author of learning resources in authentic scenarios, b) track time devoted to learn across contexts, and c) provide seamless access to learn in frequently used lifelong learning scenarios.

\item Identify barriers for digital competence in lifelong learners. This thesis has shown that the combined use of multiple device types might lead to notable differences in digital competence. The results outlined in \textbf{Chapter 8} pinpoint to differences between users with regard to familiarity with the mobile device, the operating system of the device and the complexity of the user interface as key aspects to take into account when designing technology to support lifelong learners. Additionally, the results obtained in \textbf{Chapter 9} show that usability, number of clicks, and response time, are key variables affecting the integration a mobile tool in long-term learning routines.

\item Foster reflective practice on meta-learning. All along this thesis we have explored the effectiveness of mobile notifications to foster reflection on meta-learning and to develop understanding on how to model learn the way to learn while adapting to daily life contingencies and personal priorities. Therefore, in a first step ( \textbf{Chapter 4} ) we explored the variables affecting the user when sampling experiences on mobile devices describing a classifications framework. A set of software tools were developed in order to identify user preferences with regard to the way to receive, dispatch and answer notifications on mobile devices. Later on, in \textbf{Chapter 8} we observed at the �timing�, �frequency�, and �complexity� of the notifications. The results show that students do not have a habit to see themselves as learners and to develop a professional awareness about their daily activity at work/school. On the other hand, the measures obtained envisioned better outcomes for the group of learners that were assigned with the least complex interactions on mobile devices during the reflection exercise. Based on this experience, the longitudinal study described in \textbf{Chapter 9} used an evolved version of the NFC-LearnTracker presented in \textbf{Chapter 6}, to observe variations in the �timing� and �content� of the notifications. The results reveal positive effects of tracking learning time in �time-management� skills. Variations in the channel, content and timing of the mobile notifications for self-reflection were investigated and time-logging patterns are described. These results not only provide evidence of the benefits of recording learning time, but also suggest relevant cues on how mobile notifications should be designed and prompted towards self-regulated learning.
\end{itemize}
 
\section*{Limitations of this research}
By identifying daily practices from lifelong learners to support them with ubiquitous technology this thesis has contributed towards a common understanding of how lifelong learners can explore their habits and build their autobiography as professional learners. However, several factors limited the conducted research

This research is largely based on prototypes that were evaluated in not ideal lifelong learning settings. Lifelong learners model daily learning routines based on intrinsic motivations and personal decisions taken in real circumstances. Some of the tools used in these studies were not discovered by the learners themselves, but rather offered to them from external researchers. Thus, initiatives to mentally evoke what they have learned throughout the day to turn the learning experience into a deliberate object of attention and reflection are more powerful when they come from personal inquietudes rather than from external invitations.

Some formative studies had a too short duration. Modelling one�s lifelong learning day implies long-term evaluations in which moments of the day and moments of the week are self explored by the learner. 

This research in parts used small test audiences (even below n=10), which sometimes made generalizations difficult.

Most of the work reviewed in this thesis is limited to scientific literature. Nevertheless, a big proportion of the developments from lifelong learners are not documented because they are based on personal understandings and consequently do not have scientific value.
