% To-Dos:

%\chapter*{General Discussion} % Write in your own chapter title
\chapter*{General Discussion\footnote{This chapter incorporates discussions and conclusions from several publications.}\markboth{General Discussion}{}}
\addcontentsline{toc}{part}{General Discussion}

%\vfill
%\clearpage

REFRAME  THIS PARAGRAPH IS FROM (Fischer, G. (2001). User Modeling in Human�Computer Interaction. User Modeling and User-Adapted Interaction, 11, 65�86. doi:10.1023/A:1011145532042):
The challenge in an information-rich world (in which human attention is the most valuable and scarcest commodity) is not only to make information available to people at any time, at any place and in any form, but to reduce information overload by making information relevant to the task-at-hand and to the assumed background knowledge of the users. Techniques to say the ``right'' thing include: (1) support for differential descriptions that relate new information to information and concepts assumed to be known by a speci�c user; and (2) embedded critiquing systems (Fischer et al., 1998) that make information more relevant to the task-at-hand by generating interactions in real time, on demand, and suited to individual users as needed. They are able to do so by exploiting a richer context provided by the domain orientation of the environment, by the analysis of partially constructed arti- facts and partially completed speci�cations. To say things at the ``right'' time requires to balance the costs of intrusive interruptions against the loss of con- text-sensitivity of deferred alerts (Horvitz et al., 1999). To say things in the ``right'' way (for examples by using multimedia channel to exploit different sensory channels) is especially critical for users who may suffer from some disability (Stephanidis, 2001).



\section*{Review of results}
This thesis presented the results of the conducted research and development of ambient learning displays. The reported results were structured into three parts: the theoretical foundations, formative studies, and empirical findings. An elaborated conceptual framework and an extensive literature review explored the research field and laid the foundation for further research (\textbf{Part I}). Several formative studies informed the theoretical work as well as the design and development from different perspectives (\textbf{Part II}). Following up, empirical studies then evaluated respective ambient display prototypes (\textbf{Part III}).

\subsection*{Theoretical foundations}
\textbf{Chapter 1} outlined the vision of ambient learning displays � enabling learners to view, access, and interact with contextualised digital content presented in an ambient way. This vision was based on a detailed exploration of the characteristics of ubiquitous learning and a deduction of informational, interactional, and instructional aspects to focus on. Based on these aspects the main research question was formulated: 

\begin{quote}
\textit{What are the effects of ambient information presentation on learning in a \newline situated learning context within ubiquitous learning environments?} 
\end{quote}

%Derived from this question the main research objectives were:
%\begin{itemize}
%	\item Establish the awareness for information relevant for situated learning within ubiquitous learning environments.
%	\item Examine the personal, social, and environmental sense-making process facilitated through ambient information presentation within ubiquitous learning environments.
%	\item Evaluate the situated learning support in authentic learning situations on its effectiveness for learning, especially to solve problems in context.
%\end{itemize}

To provide a theoretical foundation for the following research, relevant research findings, models, design dimensions, and taxonomies were examined. The result was a conceptual framework that defined ambient learning displays. The framework consists of parts dedicated to user and context data acquisition, channelling of information, and delivery of contextualised information framed in a learning process. Ambient systems were proposed as means of delivery. Based on the taxonomy of ambient information systems by \cite{Pousman2006b} four design dimensions were introduced, incorporated in the framework, and applied whenever designing ambient display prototypes throughout the following research. The four design dimensions are:
\begin{compactitem}
	\item Information capacity determined by the amount of information presented by the system.
	\item Representational fidelity describing how the data is encoded.
	\item Notification level depicting the degree of user interruption. 
	\item Aesthetic emphasis.
\end{compactitem}

As first step on the research agenda towards ambient learning displays an extensive review of the literature on ambient displays was conducted. The first part of the review in \textbf{Chapter 2} depicted characteristics, classified prototypical designs, and shed light on the actual use of the covered ambient displays, their application context and addressed domains as well as the type of studies conducted, including the used methodologies and evaluation approaches to measure their effectiveness and impact. The review confirmed the following main characteristics of ambient displays as defined initially by \cite{Wisneski1998a}:
\begin{compactitem}
	\item Peripheral, unobtrusive, and embedded design addressing various senses.
	\item Utilisation of subtle methods in the periphery of attention.
	\item Focus on ensuring awareness of mostly non-critical information. 
\end{compactitem}

The presented prototypes were applied in personal, public, or semi-public contexts. In a personal context the displays were closely linked to individuals (or the individuals close proximity) with high emphasis on privacy. Addressed domains included leisure activities, health related issues, or information awareness. In a public context, the displays implemented a strong environmental link with low emphasis on privacy. Addressed domains included among others consumption and conservation of natural resources. Between the two levels in a semi-public context the covered displays linked to, e.g. workplaces or classrooms. Consequently, addressed domains included group collaboration, communication, or awareness. Single prototypes were also used within educational settings.

Several articles derived principles and guidelines for the design of ambient displays, some of which were already incorporated in the conceptual framework. The classification of the presented prototypes within the framework revealed that the main source to acquire relevant information is sensor data, monitoring user activity and behaviour within the environment. The gathered information is then to a great extend channelled harnessing the learners� sense of vision. In terms of framing into a learning process, most prototypes do not go beyond the factual knowledge dimension, while addressing lower cognitive processes. This result is in line with the initial expectation to find a large number of ambient displays that simply represent information rather than supporting more complex cognitive processes.
 
The methodological analysis of the reviewed articles highlighted a plethora of methodologies and evaluation strategies used in the different types of study conducted. The whole array of available instruments was applied in the course of the complete research cycle (preparation, implementation, analysis) covering pilot studies, field studies, or a combination of both. In general, the evaluation of ambient displays was described as difficult across the reviewed articles. The following issues were identified as most critical:
\begin{compactitem}
	\item Unobtrusive collection of necessary data without additional user distraction. 
	\item Usefulness of evaluation when user interest is not stable.
	\item Evaluation of the usefulness, benefit, and comprehension.
	\item Adaptation of heuristics to evaluate usability and effectiveness.
\end{compactitem}

The literature review continued in \textbf{Chapter 3}. This second part focused on the actual use of ambient displays in a learning context. The purpose was to assess ambient displays with an explicit or implicit learning purpose and the classification of respective prototypes on the basis of an extended classification framework. The classification framework included three perspectives:
\begin{compactitem}
	\item Informational and interactional design of the prototypes.
	\item Research objectives and results of the reported empirical studies. 
	\item Deducible instructional characteristics.
\end{compactitem}

To describe the informational and interactional design, the introduced design dimensions were used. For the instructional characterisation the concept of situational awareness as defined by \cite{Endsley2000} with the three levels perception, comprehension, and projection as well as the research variables of interest when providing instructional feedback by \cite{Mory2004} were used. Thus the incorporated classification criteria were: level of situational awareness, complexity, timing, error analysis, and learning outcome.

The classification of the ambient display prototypes according to the introduced design dimensions showed that the majority of prototypes handled only a low capacity of information, were reluctant when it comes to the level of notification by just making aware, utilised all available representational means from indexes to symbols, and put a medium emphasis on aesthetics. 

Next the research objectives, reported results, and findings were analysed. The majority of articles targeted basic psychological effects, such as raise, enhance, or support awareness. Only some went beyond stating to trigger changes in behaviour or give direct feedback, calling for more research on concepts like persuasion, motivation, feedback, and behaviour change to lay the foundation for learning processes supported by ambient displays. Several papers focused on an evaluation of the displays design without evaluating learning effects explicitly. The missing focus on learning effectiveness opened up a research gap towards ambient learning displays. 

The reported result can be classified into groups dealing with the user experience, functionality, design, and evaluation of ambient displays. In general, the displays were experienced positively and perceived as characterised. They fulfilled their intended functionality, especially providing information and presence awareness. To trigger changes in behaviour the displays needed to be engaging and motivating, e.g. by providing direct feedback. A minority of articles explicitly used ambient displays for learning, many more addressed learning implicitly. 

Most reviewed prototypes addressed the lowest level of situational awareness, i.e. perception. It was noted that learning outcomes involving higher cognitive process capabilities were most effectively addressed by abstract information representations with at least simple verification feedback incorporating corrective error analysis. In contrast declarative or concept learning also worked with no feedback complexity and simple confirmatory error analysis. Other conclusion drawn were:
\begin{compactitem}
	\item Abstract representations are more effective on raising awareness, foster self-regulation, or increase behavioural impact.
	\item Feedback complexity, error analysis, and learning outcome increase with the situational awareness level.
\end{compactitem}

In summary, the mapping of the corresponding prototypes with the introduced classification framework led to first general design implications, taking into account instructional characteristics. The framework proved to be suitable to draw conclusions on the effectiveness of prototypical variations in a learning context.

\subsection*{Formative studies}
\textbf{Chapter 4} described an explorative study conducted to inform the research of ambient learning displays. By asking domain experts the used concept mapping approach identified the major educational problems that can be addressed by mobile learning and clustered these problems into domain concepts that contribute to a definition of mobile learning. Although the study targeted on the mobile learning domain, the results were in a broader view also considered valuable for the ubiquitous learning domain. Especially the following identified problem statements were related to the mobile and the ubiquitous learning paradigm:
\begin{compactitem}
	\item Access to learning resources and learning opportunities without the restrictions of location, time and cumbersome equipment or facilities.
	\item Ability to discover and experiment in own context.
	\item The provision of access to knowledge in the context in which it is applied.
	\item Accessibility of information in relevant everyday life and work situations.
\end{compactitem}

 These statements were clustered in the following main problem cluster:
\begin{compactitem}
	\item Access to learning covering problems that are mainly related to the challenges of enabling learning in a mobile society.
	\item Contextual learning comprising problems that highlight the relation between learning and the context in which the learning takes place. 
\end{compactitem}

The identified educational problems and derived domain concepts reflected the claim of mobile and ubiquitous learning to enable learning across context, facilitating and exploiting the mobility of the learners. The emphasised issues mainly discussed learning activities and opportunities outside of formal settings with better contextualised and situated learning support. The results were used both as indicators for the research focus and as an instrument to validate research findings.

The results of two projects that informed the design and development of ambient learning displays were presented. The first project, presented in \textbf{Chapter 5}, elaborated and developed an infrastructure that supports energy conservation at the workplace. Therefore the infrastructure utilised existing services and included individual energy consumption information. The main idea was to make hidden consumption data visible and accessible for the people working in the building. The infrastructure implemented the following functionality in line with the presented conceptual framework, i.e. data acquisition and channelling of information:
\begin{compactitem}
	\item Inclusion of individual energy consumption information (device specific or personal level of detail.
	\item Aggregation of available information extending and enriching the overall energy consumption picture. 
	\item Sensoring and logging to measure the effectiveness in terms of energy conservation and enable the prototypical evaluation. 
\end{compactitem}

Based on the supporting infrastructure, application prototypes to access and explore the information were developed. The prototypes were classified according to the presented classification framework. Regarding the instructional capabilities the developed prototypes went beyond the mere level of information perception. Instead, the addressed situational awareness demanded at least the comprehension of the available informational cues. To use the application prototypes efficiently even demanded to forecast and estimate the implications of the personal consumption behaviour. In terms of the used feedback characteristics, the prototypes provided simple verification feedback that could be more elaborated on demand. The timing was immediate, although the delivery of information was not happening in real-time due to technical restrictions. The feedback intended to convey at best relational rules as learning outcome, while not going beyond the confirmatory analysis of errors. 

Besides measuring the effectiveness of the prototype, an informative study, a comparative study, an user evaluation, and a design study were conducted. The results indicated the general interest in the topic as well as the usefulness of the prototype in terms of estimation and concern about the individual energy consumption. Participants were especially interested in investigating and adapting their consumption patterns accordingly. The design study revealed a preference towards an indexical representational fidelity.

The second project, presented in \textbf{Chapter 6}, implemented a pervasive game to increase the environmental awareness and pro-environmental behaviour at the workplace. Based on a discussion of the theoretical background and related work, the game design and game elements were introduced. The presented evaluation results showed that a pervasive game is a promising approach to involve employees actively in the energy conservation of an organisation. Incentive mechanisms, such as rewards in form of digital badges, were less important. All game elements that contributed to knowledge building or that involved participants in problem solving or the development of own ideas (activity, action, challenge) had more influence on pro-environmental consciousness and pro-environmental behaviour. When asked about improvement suggestions, participants called for an engaging game design, multiplayer options, and personalised feedback.

\textbf{Chapter 7} then described a lecture series that summarised the theoretical foundations. Furthermore, the chapter reported on a participatory design study conducted in the course of the lectures. The presented results showed a variety of usable ambient display types, possible learning scenarios, and specific design proposals towards ambient learning displays. The participants described different ambient display types, whereas the majority either utilised embedded display screens or billboards, converted existing technical appliances of daily use, or harnessed mainly visual appliances like glass, windows, or mirrors. The participants had difficulties describing concrete learning scenarios and respective ambient learning display design. The scenarios described by the participants mainly had one of the following objectives:
\begin{compactitem}
	\item Increase awareness of contextual information.
	\item Provide feedback on user action.
	\item Support the learning of languages or psychomotor skills.
\end{compactitem}

\subsection*{Empirical findings}
This first empirical study into the research and development of ambient learning displays was presented in \textbf{Chapter 8}. The first part of the study reported an intervention to initiate environmental learning and facilitate pro-environmental behaviour. The purpose was to examine the impact of ambient learning displays on energy consumption and conservation at the workplace, more specifically the evaluation of learning outcome and behaviour change. For the experimental treatments, prototypes were varied on two design dimensions, namely representational fidelity and notification level. The research questions were:
\begin{compactitem}
	\item Does the design of an ambient learning display influence the environmental learning outcome?
	\item Do the ambient learning display prototypes deployed lead to pro-environmental behaviour change?
\end{compactitem}

The first hypothesis stated that using interruptive notification and symbolic representation should result in a significantly larger environmental learning outcome than using change blind notification and indexical representation. The results did not show evidence to support this hypothesis. The group with the interruptive and symbolic prototype design had the largest outcome, but the design variations had no significant influence on this. 

When investigating the environmental learning construct�s single components, namely the participants� environmental awareness, confidence, knowledge about consumption, as well as concern and conservation attitude, some supporting evidence for the hypothesis were found. The group with the interruptive and symbolic design showed the largest gain in confidence and awareness, indicating that this design lowered the awareness need and built up confidence to estimate the actual consumption and conservation potentials. On the other hand the group with the change blind and indexical design showed the largest gain in knowledge, suggesting that this design supported the examination and comprehension of the provided consumption information, saving tips, and conservation potentials. The group exposed to the change blind and symbolic design showed the largest gain in concern and conservation attitude. 

The second hypothesis stated that independent of the display�s design variation the sole deployment of ambient learning displays should facilitate pro-environmental behaviour change. Again there was no supporting evidence that the prototypes have an influence on the conservation activities performed. The results suggested that the prototypes deployed even had an opposite effect. Testing the single component mean differences across all participants showed that the deployed prototypes significantly influenced awareness and knowledge. So the prototypes did not facilitate pro-environmental behaviour but at least helped to examine, comprehend, and lower the awareness need.

The second part of the study, presented in \textbf{Chapter 9}, then focused on the interaction between ambient displays and users. The main purpose was to examine the general user attention towards ambient displays as well as the influence of different display designs. The study combined non-intrusive evaluation techniques as a quantitative approach to measure user attention with qualitative measurement of user perception and comprehension. The hypothesis was that variations in the display design affect the user attention towards the display. As additional evidence knowledge transfer was incorporated assuming that a better knowledge transfer is another indicator for an effective attention design. The criteria of interest were noticeability, disruption, comprehension, appeal, and relevance. The assumption was that these criteria have a direct influence on the knowledge transferred. 

The presented results showed a high degree of user interest in the displays over time. The highest attention rate was measured during the first days of the study, while it peaked again in the middle and at the end of the study. The user interest did not stabilise in the course of the study calling for a longer evaluation period. The results were inconclusive regarding the initial hypothesis on the effectiveness of the chosen representational fidelity and the level of notification. However, the results suggested that the chances are higher to get the user attention when designing ambient displays with easy to grasp information and a sensible but not demanding level of notification.

Besides looking at the quantitative attention data the study tried to support it�s claims with additional qualitative data. Supporting the conclusions drawn from the actual attention measurement, the reported results were inconclusive regarding the reported disruptiveness and comprehension of the displays. Participants took more notice of an interruptive display presenting symbolic information. At the same time they also felt more disrupted by it. These factors influenced each other. The study results also revealed other potential relationships, especially the high impact of an appealing information visualisation. The presented information was perceived more useful and relevant, while the information display was considered less disruptive. The noticeability of the display improved the comprehension of the information presented.

Following the initial hypothesis, this should have also affected the knowledge transferred and thus provided another indicator regarding the user attention. Again the results were inconclusive regarding the effectiveness of the chosen representational fidelity and level of notification. Nevertheless they provided evidence for the importance of providing comprehensible and relevant information. Thereby the perceived usefulness and relevance of the presented information acted as a trigger for the knowledge transfer. The result called for a contextualisation of the information presented.

Finally, the second empirical study into the research and development of ambient learning displays was presented in \textbf{Chapter 10}. The study reported an intervention to investigate previously identified research challenges on the evaluation and use of ambient displays in a learning context with the objective to gain insights into the interplay between display design, user attention, and knowledge acquisition. A display prototype corresponding to the main ambient display characteristics was designed, applied in a controlled authentic setting, and evaluated accordingly. The prototype conveyed indexical information and was enhanced with a custom-built sensor to measure user attention and trigger interruptive notifications. Using an experimental research design, a treatment group exposed to this attention-aware display design was compared to a control group. \\

The research questions were whether the attention-aware display design could:
\begin{compactitem}
	\item capture the user�s focus of attention,
	\item influence the knowledge gain, and
	\item meet the general ambient display requirements.
\end{compactitem}

The first hypothesis stated that an attention-aware display design attracts the attention earlier and retains the attention longer. The results showed clear evidence to support this hypothesis. The group exposed to the attention-aware display design reached the second attention level significantly earlier than the control group, on average in almost half of the time. The attention was also retained longer in the treatment group. The participants paid significantly more attention than the control group, on average more than twice as much.

The hypotheses regarding the second research question were that there is a correlation between user attention and knowledge gain in general and that an attention-aware display design can facilitate this gain. The results showed evidence to support both hypotheses. There was a positive relation between the amount of attention a participant paid towards the display and the participant�s score in the knowledge test. Overall when participants paid more attention, their knowledge test scores were higher. Even taking prior knowledge into account the group exposed to the attention-aware display design still scored significantly higher than the control group. Overall the attention-aware designed display using indexical representation and interruptive notifications attracted and retained attention in such a way that the acquisition of knowledge (i.e. the comprehension of the presented information) was effectively facilitated.

The last hypothesis then stated that there would be no difference in meeting the general ambient display requirements caused by the design. The results only partly showed evidence to support this. The group exposed to the attention-aware design rated the display significantly more distracting than the control group. When it comes to the perceived mental effort required for filling in the questionnaire while paying attention to the display, no significant difference between the groups was found. This lack of significance supported again the hypothesis. The hypothesis was further supported when examining the number of shifts between the periphery and the focus of attention. Due to the notifications the treatment group shifted more but the difference between the groups was not significant.

\section*{Limitations of this research}
Several factors limited the conducted research and development of ambient learning displays, i.e. the ability to answer the research questions effectively, achieve the research objectives, or provide evidence for hypotheses. These factors can be structured into design limitations and limitations related to evaluation. Overall the chosen application domain has of course a major impact on the learning conditions in general and the design and evaluation of respective ambient learning displays specifically. Authentic and situated learning usually occurs when learners are strongly related to the placement they are active in and at the same time far away from traditional (mostly formal) learning capabilities they would usually make use of. The characteristics of the current placement and the requirements of the learners have an influence on the assumptions the learners may have, the conditions they may find in situ, as well as technical constraints of the settings.

In terms of design limitations, one issue is the importance and influence of an aesthetically pleasing design especially when heading for an end-user product. The emphasis on aesthetics is one of the design dimensions derived from the taxonomy of ambient information systems by \cite{Pousman2006b}. This dimension was mostly ignored throughout the conducted research. The reason for that was the focus on evaluating the effects of ambient information presentation on learning and learning support rather than actually developing end-user products. In the context of this research putting too much emphasis on aesthetic display design was not feasible, but needs to be considered when applying the outcomes into practice.

More limitations regarding the design can also be identified in relation to the other design dimensions (i.e. information capacity, representational fidelity, and notification level). The information presented by the developed prototypes was mainly limited to factual knowledge with some implications for the evaluation, which did not go beyond the simple recall of the acquired information. Addressing also the conceptual or procedural knowledge dimensions and/or more complex cognitive process dimensions should improve this. As described within the presented conceptual framework, the revised taxonomy of educational objectives by \cite{Anderson2001} describes and relates these dimensions. The second empirical study was a first step in this direction. Furthermore, it can be argued that the interruptive notification design used in the empirical studies was too obtrusive (even distracting) and that the observed effects are merely a result of this. Of course, the prototypes were designed in such a way that their experimental variations depicted borderline manifestations of the manipulated ambient display characteristics. Reproducing the observed effects, future designs should gradually decrease the level and frequency of notification. However, finding the right balance between obtrusiveness, appeal, and effectiveness remains one of the biggest design challenges. The reported results can only provide rough design guidelines that need to be reconsidered in any given context.

Within the empirical studies, limitations regarding the measurement and analysis of user attention towards ambient displays were revealed. In the first empirical study the quantitative approach using sensor data was no reliable measurement of user attention. Single users could not be identified and thus no statistical methods were applied for the analysis. The data only presented a rough estimation of the actual user attention. The added qualitative approach did not solve this problem completely. The gathered questionnaire data was not a valid measurement of user attention. Single users could be identified and thus statistical methods were applied to analyse the data. Still the used questionnaire was no conclusive inventory of user attention. Nevertheless the combination of quantitative and qualitative measurement provided already a more holistic view on user attention. In the second empirical study the effects of these limitations were reduced. Again sensor data was used as a measurement of user attention, but the experimental setting allowed identifying single users. Further improvements of the custom-built attention sensor and more detailed logging capabilities made the measurement reliable and allowed detailed statistical analysis. Another limitation is also related to this measurement. The user attention was measured conceptually assuming that a user pays attention to the display and the information presented whenever looking at it. Using the custom-built attention sensor this process is even more abstracted to a level where facing the display frontally equals looking at the display and thus paying attention. Although this broad abstraction is confirmed by previous and related research, there are much more reliable measures available. A proposed solution was eye tracking. Even though eye tracking becomes less intrusive with latest technical developments (e.g., mountable remote trackers, tracking glasses), it is still more intrusive compared to the sensor method used in the empirical studies.

Finally, the evaluation of ambient learning displays in the conducted empirical studies has also several limitations. The main issue is related to the experimental setting used. As noted in the theoretical foundations, the evaluation of ubiquitous scenarios in laboratory settings is self-contradictory. While ubiquitous computing and the derived ubiquitous learning scenarios are characterised by the �anywhere, anytime� paradigm, laboratory settings per se exclude these features as they postulate the full control of all confounding variables. Evaluation techniques need to take into account the current context, environment, and conditions the user is experiencing within the situation that is observed. \cite{Kaikkonen2005} also discussed this tension. While lab settings offer a context without the danger of uncontrollable external variables, they have also been criticised as having a very low ecological validity. On the other hand, field settings suffer from many external variables that can influence the results of an experiment while being highly authentic and therefore offering the best ecological validity possible. Thus for both presented empirical studies the core question was how to evaluate in realistic settings controlling confounding variables? 
The experimental setting used in the second empirical study proved to be a good compromise between internal validity and ecological validity.

\section*{Implications and future research}
The presented vision of ambient learning displays highlighted the challenges and explored the possibilities that lie in the convergence of mobile and ubiquitous learning in combination with the utilisation of contextualised digital content as valuable resources to support learning. The empirical findings delivered new scientific insights into the authentic learning support in informal and non-formal learning situations. The project investigated if there is a measurable benefit utilising ambient information presentation for a contextualised learning support within ubiquitous learning environments. The presented results of the conducted research and development entail several practical implications especially when designing and evaluating ambient learning displays. 

Examining existing prototypical ambient display designs in related research work revealed that for acquiring relevant contextual information the increasing amount of sensors available on a personal level and within the environment are exploited to monitor user activity and behaviour. The gathered information is then to a great extend channelled back harnessing the learners� sense of vision. Here the great potential of addressing multiple senses (especially in a learning context) is left unexploited. Important senses, such as hearing and haptic, are clearly underrepresented across the reviewed prototypes and require deeper investigation. Inspecting the use of ambient displays for learning revealed that the majority of prototypes address learning only implicitly by raising, enhancing, or supporting awareness, changing behaviour, giving feedback, providing assistance and guidance, or just by presenting information. More effort needs to be put into research addressing learning explicitly. In terms of framing into the learning process, prototypes designed and developed with explicit learning purpose need to go beyond the factual knowledge dimension addressing solely lower cognitive processes. Beside that, effective means to evaluate learning need to be identified and applied. This is one of the crucial aspects towards ambient learning displays. To go beyond the mere goal to support awareness and lay the foundation for learning processes supported by ambient displays, more effort needs to be put into research on concepts like persuasion, motivation, feedback, and behaviour change.

The mapping of existing prototypes with the introduced classification framework led to first general design implications, taking into account the instructional characteristics considering concepts like situational awareness and feedback. The reviewed ambient display prototypes could be described with and classified within the used taxonomy. This illustrates that the taxonomy is already well elaborated and does meet the requirements of an integrated framework for ambient learning displays. Although the framework as it is proved to be suitable to draw conclusions on the effectiveness of prototypical variations in a learning context, several research gaps and shortcomings have been revealed. Especially the impact of a direct interaction on learning and the motivation to learn has not been investigated. Also, the relation to the cognitive processes of learning and the role of changing interaction modalities accordingly lacks in-depth research. Connected to that but more an instructional question is the underutilisation of the displays� ability to move between the users� periphery and focus of attention, as most prototypical designs stay within one or the other. It is also questionable if the type of instructional feedback specified is sufficient to cope with the changed handling of information and interaction modalities offered by pervasive technologies such as ambient displays. Other types of feedback might be more efficient. The effect of location-based or contextualised feedback is a yet unexplored research direction in the feedback literature for which ambient learning displays can play an important role in the future. The contextualisation component is not mapped and explored sufficiently within the framework, although the correlation of context, display effectiveness, and chosen design is obvious.

The empirical findings helped to understand and estimate the potential of the introduced ambient learning display prototypes. The studies focusing on user attention revealed that the effectiveness of ambient displays highly depends on surrounding conditions, e.g. differences in the frequency of users passing-by during the day. Existing daily routines need to be considered and the display design should be adapted accordingly. Several guidelines for an effective attention-aware display design can be derived. Successful display designs need to be contextualised and should not go beyond �just the right� level of obtrusiveness while providing glanceable information. A possible bias is the novelty effect that accounts for outstanding user attention shortly after deployment and then levels off quickly. To reach the next phase and finally get the desired message through, users need to be intrigued and motivated. The intended transfer of knowledge is initiated when the presented information is both relevant and appealing. Finally, comprehensibility facilitates the process. This also confirms for instance \cite{McCrickard2003} stating that the main challenge for an effective ambient display design is the right balance of attention and usability. How to deal with this trade-off is mainly influenced by the intended use of the display. Future research will then focus on the coherent attention-aware and contextual design of ambient displays.

In terms of the observed effects, the approaches offered a promising way to increase awareness, initiate behaviour change, and support knowledge acquisition. Learning was addressed explicitly. Although the variation on the prototypes� representational fidelity and notification level proved to be inconclusive in the first study, the results revealed different effective design strategies depending on the purpose of the educational initiative. To form habits the results called for a provision of (direct) feedback reflecting individual behaviour and the use of alternative motivational designs, such as gamification \citep{Werbach2012}. It became clear that confounding variables need to be somehow controlled and generalisability reduced to reach the desired goal and evaluate learning effectively. Consequently, in the second study the main purpose was then to design a noticeable, unobtrusive, and comprehensive display that is capable to retain user interest over time, evaluate this design in an authentic setting controlling confounding variables, and gain insights into the interplay between display design, user attention, and knowledge acquisition. The attention-aware display design attracted and retained user attention more effectively and significantly facilitated the acquisition of knowledge. The design was evaluated in an authentic setting using valid methods. Considerably more work will need to be done to determine the cognitive transfer of knowledge on the long term. Furthermore, the sustainability of the observed attention effect needs to be investigated in detail.

Towards ambient learning displays still some work needs to be done, wherein this thesis can be taken as basis and inspiration to go beyond. The focus needs to be on the development of new display types addressing the whole range of senses as well as the utilisation of existing already embedded displays. Regarding learning scenarios theoretical concepts like situational awareness and feedback need to be incorporated to shape learning experiences so far not touched upon by ambient displays. The actual design of ambient learning displays in compliance with ambient display characteristics remains challenging but not impossible.

\subsection*{Designing an ambient learning display}
Based on the results of this thesis an ambient learning display can be designed in the following way. Certainly, the display design follows the original definition of ambient displays, but slightly breaks with certain aspects to find the right balance between unobtrusiveness, appeal, and effectiveness within the learning context. In addition to whatever is happening in the learner�s main focus of attention the embedded display enriches the environment with digital information using an appealing representation. This information needs to be contextualised, comprehensible, and relevant. To raise the learner�s awareness of something, the display makes use of an interruptive design showing easy to grasp and glanceable information. To address higher cognitive process dimensions, such as knowledge acquisition and transfer, the display presents rather abstract information and makes use of a less interruptive design. This process can be facilitated with at least a simple verification feedback on the performance or behaviour of the learner and the correction of possible errors. In the long term the display needs to provide elaborated feedback or use an appropriate incentive mechanism to engage and motivate the learner further and not loose the learner�s interest and attention. In general the display follows the basic rationale that the more attention a learner pays, the more knowledge can be transferred. Therefore, the learner attention needs to be balanced in a way that the noticeability and disruption of the display are adjusted to the targeted comprehension and relevance. This can be done with a careful attention-aware design involving the display, the learner, and the environment. Such a design then also enables the continuous non-intrusive evaluation of the various aspects, which can again be used to create a closed feedback loop.