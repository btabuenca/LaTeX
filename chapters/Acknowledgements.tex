% Acknowledgements

% To-Dos:
%
% Footnote
% Figures
% Tables
% References

\chapter*{Acknowledgements\markboth{Acknowledgements}{}} % Write in your own chapter title
\addcontentsline{toc}{chapter}{Acknowledgements}

For their continuous support I would like to thank my family and friends, especially my girlfriend, Jessica Biermann, for allowing me to start this journey and supporting me to finish it, as well as my parents, Marlene B�rner and Hans-Joachim Sternkopf, for making everything possible.

Furthermore, I would like to thank my colleagues and all the people I met while conducting this research for their inspiration, review, and support of my work. Special thanks to my \textit{promotor} Marcus Specht and my \textit{co-promotor} Marco Kalz for their outstanding supervision and guidance as well as to Mieke Haemers for her organisational wisdom.

Last but not least, I would like to thank my \textit{paranimfen}, \mbox{Halszka} \mbox{Jarodzka} and \mbox{Bernardo} \mbox{Tabuenca}, for their broad support and friendship, my \textit{promovendus} companion, \mbox{Sebastian} \mbox{Kelle}, for his advice and the mutual cultural exchange, as well as the band Tocotronic for their creative and inspiring song titles.

\section*{Funding sources}
The research project presented in this thesis was mainly funded by the STELLAR Network of Excellence, the OpenScout project, and the weSPOT project. STELLAR is a 7th Framework Programme project funded by the European Commission, grant agreement number: 231913 (http://www.stellarnet.eu). OpenScout is a 7th Framework Programme eContent+ project funded by the European Commission, grant agreement number: 428016 (http://www.openscout.net). weSPOT is a 7th Framework Programme research project funded by the European Commission, grant agreement number: 318499 (http://wespot-project.eu).

The projects reported in \textbf{Chapter 5} and \textbf{6} have been partially funded by a SURFnet innovation grant for sustainable ICT solutions and partially by the Centre for Learning Sciences and Technologies (CELSTEC) of the Open Universiteit in the Netherlands.

As part of a subordinated small-scale study on location-based and contextual mobile learning, the study presented in \textbf{Chapter 4} was conducted in cooperation with the Learning Sciences Research Institute (University of Nottingham) within the framework of the STELLAR Network of Excellence.

The study presented in \textbf{Chapter 7} was conducted in the context of two lectures. Special thanks to all participants of the SIKS advanced course on �Technology-Enhanced Learning� and the master students attending the guest lecture on �Ambient Learning Displays� as part of the RWTH master course on �Advanced Learning Technologies�.

Finally the study presented in \textbf{Chapter 10} was partly funded by the European Regional Development Fund (ERDF), regions of the Euregio Meuse-Rhine (EMR), and the participating institutions within the implementation and research project �EMuRgency - New approaches for resuscitation support and training� (EMR.INT4-1.2.-2011-04/070, http://www.emurgency.eu) under the INTERREG IVa programme.