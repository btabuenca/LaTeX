% To-Dos:

%\chapter*{General Introduction} % Write in your own chapter title
\chapter*{General Introduction\footnote{This chapter incorporates abstracts and introductions from several publications.}\markboth{General Introduction}{}}
\addcontentsline{toc}{part}{General Introduction}

%\vfill
%This chapter incorporates abstracts and introductions from several publications.
%\clearpage

\begin{quote}
\textit{Ubiquitous technology. Si vas a Calatayud, pregunta por la dolores.}
\flushright{(Lola Dolores, 1991)}
\end{quote}

Describe here what is ubiquitous technology for you. 

\textit{Si vas a Calatayud, pregunta por la dolores.}
\flushright{(Lola Dolores, 1991)}

Describe here what means lifelong learners 


\section*{Outline of the thesis}
The thesis is structured into three parts: theoretical foundations, formative studies, and empirical findings. An elaborated conceptual XXXXXXXXX XXXXXXXX XXXXXXX XXXXXX and an extensive literature review explore the research field and lay the foundation for further research. \textbf{Chapter 1} starts with outlining the vision of ambient learning displays....

\textbf{Chapter 2} then presents r....

The review continues in \textbf{Chapter 3} analysing work ....

Several formative studies inform the theoretical work as well as the design and development from different perspectives. \textbf{Chapter 4} first of all introduces ...

\textbf{Chapter 5} presents ....

A pervasive game to increase the environmental awareness and pro-environmental behaviour at the workplace is presented in \textbf{Chapter 6}. ....


\textbf{Chapter 7} summarises ....

Following up the theoretical work and the formative studies, empirical studies then evaluated ambient learning display prototypes. The first study presented in \textbf{Chapter 8} reports an i....

Related to this \textbf{Chapter 9} then presents an a....

Finally the second study presented in \textbf{Chapter 10} reports a...

The thesis concludes with a \textbf{General Discussion} reviewing all reported results and their practical implications, general limitations of the conducted research, as well as future research perspectives.