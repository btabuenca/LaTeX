% To-Dos:

%\chapter*{General Introduction} % Write in your own chapter title
\chapter*{General Introduction\footnote{This chapter incorporates abstracts and introductions from several publications.}\markboth{General Introduction}{}}
\addcontentsline{toc}{part}{General Introduction}

%\vfill
%This chapter incorporates abstracts and introductions from several publications.
%\clearpage

\begin{quote}
\textit{The most profound technologies are those that disappear. They weave themselves into the fabric of everyday life until they are indistinguishable from it.}
\flushright{(Mark Weiser, 1991)}
\end{quote}

In Mark Weiser�s vision of the computer for the 21st century, computers should be the next technology becoming �an integral, invisible part of people�s lives�. Basically, the computers would be seamlessly integrated in our life, accessible and connected via networks or somewhere around us �imbedded in walls, chairs, clothing, light switches, cars � in everything� \citep{Weiser1999}. He described this as ubiquitous computing �characterized by the connection of things in the world with computation�. And where are we now? Well, his vision is almost the reality. Although computers are not yet fully seamlessly integrated in this world, they are already highly interconnected and interwoven with our daily practice. Along with the growing connectivity, getting mobile is the trend of our time and can be seen as one intermediate stage towards ubiquitous computing.

Following these trends, it is expected that the number of mobile-connected devices will exceed the number of people on earth by the end of this year \citep{Cisco2013}. In the next stage more and more of these devices might indeed become ubiquitous � not only with and close to the people but blended into the environment. Ambient displays are one possible technical implementation within this next stage. The term originates from advertising, characterising appliances such as advertising pillars or billboards. Looking at linguistic definitions, the adjective ambient is described as �relating to the immediate surroundings of something� or �relating to or denoting advertising that makes use of sites or objects other than the established media� \citep{OxfordDictionaries2010}. The noun display is among others described as �a collection of objects arranged for public viewing�, but also as �an electronic device for the visual presentation of data or images� \citep{OxfordDictionaries2010}. Inspired by Weiser�s vision, \cite{Wisneski1998a} introduced ambient displays in the context of ubiquitous computing as �new approach to interfacing people with online digital information�, whereas the �information is moved off the screen into the physical environment, manifesting itself as subtle changes in form, movement, sound, colour, smell, temperature, or light�. Instead of demanding attention the approach exploits the human peripheral perception capabilities. The displays situated and interacting in the close proximity are an addition to existing personal interfaces in the foreground, while the user attention can always move from one to the other and back.

The described interaction approach is not new. Looking around one can find several examples that are in line with the given definition. Just take a look at the cover of this thesis. What you see at a first glance is a number of billboards trying to capture the attention of the people passing by. They all try to convey a particular message in a visually appealing way using mainly colour and light. Some also add movement and sound to become even more intrusive. These displays live in the periphery of attention. Although they are designed to make it almost impossible to ignore them, it is possible to keep them out of the focus. Still they are not completely in line with the original definition. 

Looking again at the cover, you might find another more unobtrusive ambient display - a traffic light. Already in 1868 a first prototype was installed at a busy intersection in London. The idea was to assist police officers in directing traffic, mainly consisting of pedestrians and horse-drawn vehicles. The manually operated device combined a semaphore with moveable arms and a gas lantern showing red light to signal �Stop� and green light to signal �Caution�. With the invention of the automobile, the traffic got heavier and the idea spread. Later on the lights became electric, the semaphores were abandoned, and their operation was automated. Ever since the way of signalling remained more or less the same: a red light indicates to stop, a green light indicates the possibility to cross, and occasionally a yellow or orange light indicates (as state in-between) either to prepare for the one or the other state. 

Back to the original definition also traffic lights try to convey a message visually using again mainly colour and light. They live in the periphery and use subtle changes between the various states to capture attention and get in the focus when necessary. This functionality and the contained visual metaphor have even become ubiquitous in a sense that the concept is also used in different contexts, such as food labelling. Consequently traffic lights can be seen as successful ambient display instances following the definition of \cite{Wisneski1998a} with the exception that they do not present or at least symbolise online digital information. Adding up this peculiarity it becomes apparent that there might be some merit in applying the concept also in a learning context.

\begin{quote}
\textit{The people are fundamentally, inherently mobile � they move around; they never, never would want to be leashed tight to a desk or to their home or to their office if they have a choice.}
\flushright{(Martin Cooper, 2005)}
\end{quote}

As Martin Cooper, the inventor of the cell phone, framed it, mobility is one of the basic human needs that influences all aspects of life. Accordingly, the era of mobile and ubiquitous computing challenges the way we learn with computers. Computers as learning technology disappear from the main focus of a learner�s attention and become means to an end. Instead of acting as yet another disrupting threshold in the learning process they become integrated unobtrusive facilitators. Again getting mobile is the major trend. Mobile learning focuses on learning support across contexts and learning with mobile devices. Arguably the mobile learner of today is not one that solely uses mobiles to access traditional learning materials � rather it�s a learner who is mobile and moves through different environments and occasionally stumbles upon traditional or newly designed learning opportunities and activities. Learning in this world is mostly informal, happens accidentally and in situ, and is highly contextualised. Consequently ubiquitous learning not only enables learning across context, but also facilitates and exploits the mobility of the learners instead of the technology. 

Following this approach the ability to deliver contextualised and personalised information in authentic situations fosters ambient displays as an instrument for learning. Up until now this has not been a major research focus. The design of ambient displays for learning proves to be difficult, as the technical implementations as well as the underlying instructional principles are still immature. These gaps are the starting point for this thesis � presenting the results of the conducted research and development of ambient learning displays.
\section*{Outline of the thesis}
The thesis is structured into three parts: theoretical foundations, formative studies, and empirical findings. An elaborated conceptual framework and an extensive literature review explore the research field and lay the foundation for further research. \textbf{Chapter 1} starts with outlining the vision of ambient learning displays. With a focus on the situated support of informal and non-formal learning scenarios in ubiquitous learning environments learners should be enabled to view, access, and interact with contextualised digital content presented in an ambient way. The vision is based on a detailed exploration of the characteristics of ubiquitous learning and a deduction of informational, interactional, and instructional aspects to focus on. Towards the vision essential research questions and objectives as well as a conceptual framework that acquires, channels, and delivers the information framed in the learning process are presented. To deliver scientific insights into the authentic learning support in informal and non-formal learning situations and to provide suggestions for the future design of ambient systems for learning the chapter concludes with a research agenda proposing the research project including a discussion of related issues and challenges.

\textbf{Chapter 2} then presents results from a recent literature review on ambient displays. While the main background of the authors is education and technology-enhanced learning, the chapter starts more generic with a broader view on ambient displays and their interactional, instructional, and informational characteristics. Beside depicting characteristics and classifying prototypical designs, the chapter also sheds light on the actual use of the covered ambient displays, their application context and addressed domains as well as the type of studies conducted, including the used methodologies and evaluation approaches to measure their effectiveness and impact. The chapter concludes with a discussion of the presented results emphasising the derived implications for the user when interacting with ambient displays.

The review continues in \textbf{Chapter 3} analysing work in the research field of ambient display with a focus on the use of ambient displays for situational awareness, feedback and learning. The purpose was to assess the state-of-the-art of the use of ambient displays with an explicit or implicit learning purpose and the possible classification of respective prototypes on the basis of a presented framework. This framework is comprised of theories around the educational concepts of situational awareness and feedback as well as design dimensions of ambient displays. The chapter presents results of recent empirical studies within this field as well as developed prototypes with a focus on their design and instructional capabilities when providing feedback. %The results expose that the explicit use of ambient displays for learning is not a prominent research topic, although implicitly ambient displays are already used to support learning activities fostering situational awareness by exploiting feedback. Overall ambient displays represent a technological concept with great potential for learning and the chapter facilitates a proper foundation and research questions for further research in this direction � towards ambient learning displays.

Several formative studies inform the theoretical work as well as the design and development from different perspectives. \textbf{Chapter 4} first of all introduces concept mapping as a structured participative conceptualisation approach to identify clusters of ideas and opinions generated by experts within the domain of mobile learning. Utilising this approach, the chapter aims to contribute to a definition of key domain characteristics by identifying the main educational concepts related to mobile learning. A short literature review points out the attempts to find a clear definition for mobile learning as well as the different perspectives taken. Based on this an explorative study was conducted, focusing on the educational problems that underpin the expectations on mobile learning. Using the concept mapping approach, the study identified these educational problems and the related domain concepts. %The respective results were then analysed and discussed. The core educational concepts of mobile learning identified are: �access to learning�, �contextual learning�, �orchestrating learning across contexts�, �personalisation�, and �collaboration�. The chapter is original as it uses a unique conceptualisation approach to work out the educational problems that can be addressed by mobile learning and thus contributes to a domain definition based on identified issues, featured concepts, and derived challenges. In contrast to existing approaches for defining mobile learning, the present approach relies completely on the expertise of domain experts.

\textbf{Chapter 5} presents a project that sets up to make energy consumption data visible and accessible to employees by providing dynamic situated feedback at the workplace. Therefore, a supporting infrastructure as well as two example applications have been implemented and evaluated. The resulting prototype fosters a ubiquitous learning process among the employees with the goal to change their consumption behaviour as well as their attitudes towards energy conservation. The chapter presents the approach, the requirements, the infrastructure and applications, as well as the evaluation results of the conducted informative study, comparative study, user evaluation, and design study.

A pervasive game to increase the environmental awareness and pro-environmental behaviour at the workplace is presented in \textbf{Chapter 6}. Based on a discussion of the theoretical background and related work the game design and game elements are introduced. Furthermore, the results of a formative evaluation study are presented and discussed. The results show that incentive mechanisms are less important than challenging game components that involve employees in proposing solutions for energy conservation at the workplace. Conclusions are drawn for future games and energy conservation activities at the workplace.

%Emerging from pervasive and mobile technologies, ambient displays present information and media in the periphery of the user. Thereby the displays situated and interacting in the close proximity are an addition to existing personal interfaces in the foreground, while the user attention can always move from one to the other and back. Especially the ability to deliver contextualised and personalised information in authentic situations fosters ambient displays as an instrument for learning. However the actual design of ambient displays for learning proves to be difficult, as the technical implementations as well as the underlying instructional principles are still immature.
\textbf{Chapter 7} summarises the main constituents of a lecture series on the use of ambient displays for learning and a participatory design study conducted during two consecutive lecture sessions. The results show a variety of usable ambient display types, possible learning scenarios, and specific design proposals towards ambient learning displays.

Following up the theoretical work and the formative studies, empirical studies then evaluated ambient learning display prototypes. The first study presented in \textbf{Chapter 8} reports an intervention to initiate environmental learning and facilitate pro-environmental behaviour. The purpose was to examine the impact of ambient learning displays on energy consumption and conservation at the workplace, more specifically the evaluation of learning outcome and behaviour change. Using a quasi-experimental design, the study was conducted among employees working at a university campus. For the experimental treatments, ambient learning display prototypes were varied on two design dimensions, namely representational fidelity and notification level. %The results do not provide clear evidence that the design of the displays influences learning outcome or that the displays lead to pro-environmental behaviour change. Nevertheless the sole deployment of the display prototypes eased the comprehension of the information provided and lowered the need for additional information. Thus ambient learning displays provide a promising framework in the context of environmental learning and beyond.

Related to this \textbf{Chapter 9} then presents an approach to better understand the interaction between users and ambient displays and the evaluation thereof. The purpose of the study was to examine the user attention towards ambient displays as well as the influence of different display designs in a combined approach using quantitative attention data as well as qualitative assessment methods. The study originates from the previous intervention using the same quasi-experimental design, but with a different research objective. %The results show a high degree of user interest in the displays over time, but do not provide clear evidence that the design of the displays influences the user attention. Nevertheless the combination of quantitative and qualitative measurement does provide a more holistic view on user attention.

Finally the second study presented in \textbf{Chapter 10} reports an intervention to investigate identified research challenges on the evaluation and use of ambient displays in a learning context with the objective to gain insights into the interplay between display design, user attention, and knowledge acquisition. The main research questions were whether an attention-aware display design can capture the user�s focus of attention and whether this has an influence on the knowledge gain. A display prototype corresponding to the main ambient display characteristics was designed, applied in a controlled authentic setting, and evaluated accordingly. The prototype conveyed indexical information and was enhanced with a custom-built sensor to measure user attention and trigger interruptive notifications. The study was conducted among employees working at a university campus. Using an experimental research design, a treatment group exposed to an attention-aware display design was compared to a control group. %The results provide evidence that such a display design can attract and retain attention in such a way that the acquisition of knowledge (i.e. the comprehension of the presented information) is effectively facilitated.

The thesis concludes with a \textbf{General Discussion} reviewing all reported results and their practical implications, general limitations of the conducted research, as well as future research perspectives.