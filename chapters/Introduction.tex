
\chapter*{General Introduction\footnote{This chapter incorporates abstracts and introductions from several publications.}\markboth{General Introduction}{}}
\addcontentsline{toc}{part}{General Introduction}


\begin{quote}
\textit{Lifelong learning has become a necessity for all citizens. We need to develop our skills and competences throughout our lives, not only for our personal fulfilment and our ability to actively engage with the society in which we live, but for our ability to be successful in a constantly changing world}
\flushright{\citep{EuropeanCommission2007}}
\end{quote}

Nowadays, most people change their career throughout their lives, many times independently on what they learned during their formal education period. Therefore, the necessity to continually keep our skills sharp and up to date becomes increasingly important in a rapidly changing job market. In 2002, the \cite{EuropeanCommission2000} stressed the importance of lifelong learning. Later on, they published a reference framework comprising eight competences to flexibly adapt to a rapidly changing and highly interconnected world \citep{EuropeanCommission2007}. In this thesis we aim at supporting learners to understand the way they can better learn using technology, therefore we focus on two specific competences, namely, \em learning to learn \em  and \em digital competence \em .

\em Learning to learn \em is defined as the ability to pursue and persist in learning, to organise one�s own learning, including through effective management of time and information \citep{EuropeanCommission2007}. This competence is closely bound to the concept of self-regulated learning defined as students' proactive actions aimed at acquiring and applying information or skills that involve setting goals, self-monitoring, managing time and regulating one�s effort towards learning goal fulfilment \citep{Candy1991}. 

On the other hand, \em digital competence \em involves the confident and critical use of technology to learn, work, and communicate in personal and social life. Technology is constantly shaping the way we learn, for instance, when we acquire a new device (e.g. a tablet, a more powerful smartphone), or change daily habits (e.g. commute by car, train) or change the way to commute. Lifelong learning \em is like a never-ending personal revolution\footnote{Bryant McGill post "The supreme lesson of education is to think for yourself". Voice of Reason. Available at http://bryantmcgill.com/20131203010300.html}  \em in which each individual constantly adapts learning routines according to the affordances given by the environment, available tools, occasional constrains, or previous knowledge on a specific topic. In this thesis, digital competence is underpinned by the intrinsic motivation to acquire basic skills in the use of ubiquitous technology to seamlessly retrieve, produce, present, and exchange learning resources.

This research aims not at guiding lifelong learning society towards the accomplishment of these two competences, but rather at providing cues for individual lifelong learners to foster meta-learning practice supported by technology. \citep{Biggs1985} defines meta-learning as an awareness and understanding of the phenomenon of learning itself. Hereby we conceive meta-learning activities as the increase of knowledge, motivation and self-regulated learning as a result of introspective episodes of reflection by the user to understand how he/she is learning.

Lifelong learners constantly change their learning context, location, goals, environments, and also learning technologies. Indeed, lifelong learners have to combine their professional activities with learning activities and must engage simultaneously with family times to ensure a balance of adults� responsibilities, overall wellbeing and their personal development. In this scenario a learner taking part in an online course might start the day during travel with the reading of the course textbook, continue at work joining an online discussion of a specific problem during the coffee break, and finish in the evening watching videos while laying on the sofa. These short learning episodes during one day are a representative picture of lifelong learning as a whole. In this scenario, the mobile device is probably the only artifact coexisting with the learner in all these scattered moments and contexts. Thus, this thesis provides cues for lifelong learners to explore how can they better learn in a variety of scenarios easily switching from one context to another, using technology, and their personal device as a mediator (seamless learning, \cite{Chan2006}).

Lifelong learners are typically active in several parallel learning tracks, which they have to manage over long periods of time, and must align or relate their learning activities to everyday family-leisure and working activities. Looking around you will easily identify the places where you normally learn (e.g. your desktop, laying on the sofa, commuting to work, in waiting times) or the resources you normally use (e.g. notebook, tablet, textbooks, mobile device). The cover of this book illustrates some of the technologies enriching daily physical environments e.g. Wi-Fi hotspots, NFC tagged objects, open content, Bluetooth beacons or ambient displays. A strength of lifelong learners is their intrinsic motivation to study and their continuous career development. Hence, they continuously spend the most of their time trying to understand the way they learn, avoiding technological obstacles, and scaffolding learning activities in daily physical environments. Looking back to the cover, you will be able to recognize different signs indicating the key concepts to support lifelong learners with ubiquitous technology tackled in this thesis \citep{Candy1991,Kalz2014}:

\begin{itemize}
\item Reflective practice. This thesis provides evidence that reflecting about a personal identity as (professional) learner is not a common and/or understood practice. We investigate the advantages of using mobile notifications to foster reflective practice on meta-learning to cover this gap.
\item  Self-regulated learning. Learning to learn is closely bound to the concept of self-regulated learning defined as students' proactive actions aimed at acquiring and applying information or skills that involve setting goals, self-monitoring, managing time and regulating one�s effort towards learning goal fulfilment. Therefore, this thesis pilots and evaluates different mobile tools supporting lifelong learners to set goals and track their study-time towards promoting self-regulated learning.
\item  Feedback. Lifelong learners� activities differ from one student to another depending on priorities, preferences, motivations, and definitely long term schedules. Hence, it becomes more complex for external tutoring systems (i.e. LMSs, teachers, instructional designers) to provide customized guidance. This work investigates the use of different channels (i.e. visualizations, notifications) to provide customized feedback containing learning analytics and tips for self-regulated learning. Additionally, the use of ambient learning displays \citep{Borner2013} is proposed as a further line of research to design feedback services, that can be customized to user�s preferences and integrated in different learning contexts.
\item Awareness. Lifelong learning implies making students more aware of their learning processes, as well as showing them how to regulate those processes for more effective learning throughout their lives. 
\item Open Educational Resources (OERs). The growing number of open online courses (i.e. MOOCS) and the extensive collection of learning objects stored in online content repositories, depict one of the most relevant sources of knowledge for lifelong learners. Flexibility of time schedules, high specialization of the resources and no-cost accessibility, make of OER one of the main sources for lifelong learners to cover their learning interests.
\end{itemize}

\section*{Outline of the thesis}
\textbf{Chapter 1} presents the results from a survey performed in a sample of 147 lifelong learners with the aim to understand how adults learn with mobile devices, and to recognise patterns to support them with technology. These patterns capture the context in which lifelong learners are willing to learn using their mobile devices, i.e. time of the day, day of the week, duration of the session, frequently used physical spaces, and type of learning activity.

OERs represent one of the main sources for lifelong learners to cover their learning needs. In \textbf{Chapter 2}, trends in the creation, publication, discovery, acquisition, access, use and re-use of OER on mobile devices are explored in a literature review. From the content providers side, this chapter presents the results obtained from a survey performed in a sample of 23 educational repositories hosting more than 1,500,000 educational resources, in which they were prompted to report their current and expected features to support access from mobile devices. Existing mobile tools and best practices to facilitate access to educational resources stored in content repositories are highlighted.

Near Field Communication (NFC) technology has become increasingly predominant in ubiquitous computing, facilitating the linkage of digital content to physical environments with zero-clicks interactions. In \textbf{Chapter 3}, a review of scientific literature in which NFC has been used with the purpose to learn is presented. These results are classified according to the application field in which they were implemented as well as the type of interaction they feature. Additionally, an ecology of resources to facilitate video casting from mobile devices using NFC interaction is presented and prototyped.

Mobile devices play an important role tracking lifelong learners� daily activity since they are always carried around, in every moment of the day and in every location. \textbf{Chapter 4} proposes using personal mobile devices to sample learning experiences throughout the day as an approach to log learning reports in context. A classification framework for sampling learners� preferences on mobile devices is presented, and a mobile application for sampling of experiences is piloted. Both framework and formative study imply an important scaffold for lifelong learners to identify productive times during the day with mobile technologies.

In Chapter 2 we reviewed features facilitating mobile access to OER by lifelong learners from the consumer perspective. In this thesis, we also explore lifelong learners from the producer perspective. Mobile authoring tools enable learners to foster universal access to OERs not only providing channels to share, remix, or re-contextualize these, but also capturing the context in situ and in time. As a further matter, authoring educational resources in a mobile context is an authentic experience in which any user can author content inspired by real in situ experiences and reflections. In \textbf{Chapter 5}, the main barriers for mobile authoring of OERs are identified and 10 key shortcomings are identified. Based on these findings and the lessons learned in \textbf{Chapter 2}, a mobile environment to author educational resources is designed and evaluated in a formative study.

Lifelong learners� activities are scattered along the day and in different locations, making it difficult to have an account on how much time is devoted to personal learning goals. \textbf{Chapter 6} presents the NFC-LearnTracker, a mobile tool supporting the user to introspect his autobiography as a learner to identify successful learning patterns. This mobile tool binds sensor tags to self-defined learning goals, tracks time devoted to learn, and monitors learning analytics with chart visualizations. This work demonstrates a suitable approach for lifelong learners to bind scattered learning activities keeping them in a continuing learning flow. The NFC-LearnTracker is released under open source code to facilitate its extension to any NFC tags manufacturer.

A challenge for technology-enhanced learning research is not only to make information available to students at any time, at any place, and in any form, but also to provide the right feedback at the right time in the right way. The proliferation of sensor technology facilitates the scaffolding of smart learning environments. \textbf{Chapter 7} extends the tool presented in \textbf{Chapter 6} with an ecology of technologies comprising Bluetooth Low Energy, Arduino microcontroller and NFC, orchestrated in the context of a desktop-based learning environment to provide smoothly integrated and customized feedback, using ambient learning displays.

Nowadays, smartphone users are constantly receiving notifications from applications that provide feedback, as reminders, recommendations or announcements. Nevertheless, there is little research on the effects of mobile notifications to foster reflective practice on meta-learning. Therefore, \textbf{Chapter 8} covers this gap presenting the results from two studies: 1) a formative study offering a daily reflection and reporting exercise about their learning experience during the day; 2) an experiment inviting students to reflect in-action. On the one hand, the results from the first study show that students do not have a habit to see themselves as learners and to develop a "professional" awareness about their daily activity at work/school. On the other hand, the results show higher knowledge gain for the students assigned with the least complex mobile interactions

\textbf{Chapter 9} puts into practice the LearnTracker presented in \textbf{Chapter 6} as well as the framework described in \textbf{Chapter 4} to extend the research on the benefits of logging learning activities on personal mobile devices. A longitudinal study explores the effects of tracking and monitoring time devoted to learning with a mobile tool in self-regulated learning. Graduate students from online courses used their mobile devices to track how much time they devoted to learning over a period of two to four months. Repeated measures of self-regulated learning and reliability of time management were taken along the course. Variations in the channel, content and timing of the notifications are investigated and time-logging patterns are described. These results not only provide evidence of the benefits of recording learning time on time management skills, but also suggest relevant cues on how mobile notifications should be designed and prompted towards self-regulated learning of students enrolled in online courses.

The thesis concludes with a General Discussion reviewing all reported results and their practical implications, general limitations of the conducted research as well as future research perspectives.